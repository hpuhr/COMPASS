\section{Using the AppImage}

To summarize, an AppImage is a form of binary distribution in one complete package, which does not require installation of any libraries or alteration of the operating system. 
Please refer to \url{https://appimage.org/} to get additional information. \\

To obtain the COMPASS AppImage, download the latest version from \url{https://github.com/hpuhr/COMPASS/releases}. \\

Before starting, it has to be made executable (once after download) using the following command:
\begin{lstlisting}
chmod +x COMPASS-x86_64_release.AppImage
\end{lstlisting}

This is it. The application can then be run using the following command:
\begin{lstlisting}
./COMPASS-x86_64_release.AppImage
\end{lstlisting}

The following points should be considered:

\begin{itemize}  
\item The AppImage should run under any Linux distribution of similar date to Debian 9 (released July 2017) or later, but no guarantees can be made.
\item To the author's knowledge, running in virtualized operating systems is possible in some solutions, but requires additional setup (GPU acceleration).
\item Using remote desktop using VNC is slightly tricky to set up
\begin{itemize}  
\item But a few users have reported VirtualGL + TurboVNC works well
\item Connect using "vglconnect user@remotehost"
\item Run AppImage using "vglrun ./COMPASS-version.AppImage"
\end{itemize} 
\item Geographic View rendering is performed according to the local graphics card and driver, and might be limited by their capabilities.
\end{itemize}
