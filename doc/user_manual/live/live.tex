\chapter{Live Mode}
\label{sec:live_mode}

There exist 3 application modes:
\begin{itemize}
\item 'Offline': Default mode, for offline analysis
\item 'Live: Running': Provides a live display of network data, while also importing the ASTERIX data into the database
\item 'Live: Paused': Provides offline-like analysis, while caching ASTERIX network data and allows resuming into 'Live: Running' without loss of data
\end{itemize}
\ \\

\begin{figure}[H]
 \center
    \includegraphics[width=8cm]{figures/modes.png}
  \caption{Application Modes}
\end{figure}


The application switches in Live mode when ASTERIX data is imported from the network, as described in \nameref{sec:ui_import_asterix_network}. \\

Please note that the data sources have to be correctly set up for reading network traffic, as described in \nameref{sec:configure_datasources_table_content}. \\

When that Live mode is enabled and the correct network lines are setup (and active) the main window is shown as follows. 

\begin{figure}[H]
  \hspace*{-2.5cm}
    \includegraphics[width=19cm]{figures/live_mode.png}
  \caption{Main Window in Live Mode}
\end{figure}

In Live mode, most application components are the same as in Offline mode (although some are deactivated), but in the main status bar the Live mode is indicated. Using the 'Pause' button the 'Live: Paused' mode can be entered and the 'Stop' button allows stopping the network recording and returning to Offline mode.

\section{Data Sources \& Processing}

For each data source, the defined lines are shown (as L1 ... L4 buttons). If no data is received (on a define Line), the respective button is disabled (greyed out). If data is received over a line, it will become active and the received data stored in the database. A \textbf{update interval} needs to be specified for the respective data source, then the line button background color will become green if current data was received during the last 2 update intervals. \\

The line button can also be toggled - if a bold border is shown the data from the respective line stored not only in the database but \textbf{also} in main memory (RAM). The most recent 5 minutes of data are kept in main memory (RAM), and can be visualized in the existing Views. \\

Please \textbf{note} that currently only the Geographic View displays data in live mode, for performance reasons the other Views are inactive.

\section{Geographic View}

In Live mode, the Geographic View automatically shows the same elements as with the time filter, and the main components are the same as in Offline mode (although some are deactivated).

\begin{figure}[H]
    \hspace*{-2.5cm}
    \includegraphics[width=19cm,frame]{figures/geo_live_mode.png}
  \caption{Geographic View in Live Mode}
\end{figure} 

In the time filter elements, at maximum 5 minutes of past data is available. In a 1 second update newly received network data is shown, and the labels updated (if automatic labeling is enabled). \\

Using the time scrollbar, past data can be inspected, which is kept until it becomes outdated (older than 5 minutes).

When clicking the \includegraphics[width=0.5cm,frame]{../../data/icons/right.png} button, the displayed time window will again follow the most recent time. This can also be achieved by moving the scrollbar to the most right position.

\subsection{Online/Offline Display}

To indicate if the displayed data is the currently most recent or from a previous point in time, the Geographic View can display colored text (by default "Online" or "Offline"). After creation of the OSView and closing the application, this feature can be enabled using the configuration options listed in \nameref{sec:live_online_offline}.

\begin{figure}[H]
    \hspace*{-2.5cm}
    \includegraphics[width=19cm,frame]{figures/geoview_offline_status.png}
  \caption{Geographic View Overload Message}
\end{figure} 

The online state text is shown if the application is in Live:Running mode and the most recent data is displayed. In all other states (other application mode or previous time window displayed) the offline state text is displayed.


\section{Overload Detection}

The ASTERIX Import task as well as the Geographic View can detect and handle overload situations. If the real-time processing is not possible with the used workstation, display latency builds up - resulting in ASTERIX data queuing up. \\

If such a latency is larger than 3 seconds,  existing Geographic Views skip drawing until re-synchronization is possible (display latency $<$ 3s). During such an overload, the Geographic Views show a red overload message on the bottom right, with the display latency in seconds. After such an overload, no loss of data occured and all data is displayed. \\

\begin{figure}[H]
    \hspace*{-2.5cm}
    \includegraphics[width=19cm,frame]{figures/geoview_overload.png}
  \caption{Geographic View Overload Message}
\end{figure} 

If the latency exceeds 60 seconds, ASTERIX data is skipped - therefore not decoded, added to the database, or displayed.

\section{Auto-Resume From Live:Paused}

If the application is switched to the Live:Paused mode, an automatic resume function is activated to ensure that the application is not inadvertently kept in this mode for too long. In its default settings, every 60 minutes a dialog is opened which allows user interaction to stay in the Live:Paused mode. If no action is taken, after 1 minute an automatic resume in the Live:Running mode is performed. \\

This feature can be configured using the configuration options listed in \nameref{sec:live_paused_resume}.

\section{Long-Term Running}

The application is optimized for long-term running, and can be run continuously. During Live mode, data older than 1 hour is removed from the database. \\

In this use case, usage of a rotating log file (as described in \nameref{sec:appendix_logging_rotate}) is recommended.

\section{Configuration Options of Interest}

Since the Live mode may be set up for users which only use COMPASS for limited use cases, a number of configuration options were added to allow for a simpler usage. The listed files exist in the configuration folder in the 'conf/default' subfolder.

\subsection{Disable Adding/Removing Views}

In the file 'compass.json' the following parameters can be set:

\begin{lstlisting}
...
        "disable_add_remove_views": false,
...        
\end{lstlisting}

\subsection{Disable Switching Live to Offline Mode}

In the file 'compass.json' the following parameters can be set:

\begin{lstlisting}
...
        "disable_live_to_offline_switch": false,
...        
\end{lstlisting}

\subsection{Change Cache or Database Time Duration}

Per default, the RAM cache is limited to 5 minutes, while the database content is limited to 60 minutes. To change these durations (if workstation performance allows for it), the following values can be set in the file 'db\_content.json' (values in minutes):

\begin{lstlisting}
...
        "max_live_data_age_cache": 5,
        "max_live_data_age_db": 60,
...        
\end{lstlisting}

\subsection{Geographic View Online/Offline Display}
\label{sec:live_online_offline}

In the 'views.json' configuration file, in the respective Geographic View section, the following parameters can be set:

\begin{lstlisting}
...
        "offline_text": "Offline",
        "offline_text_color_str": "#FF0000",
        "online_text": "Online",
        "online_text_color_str": "#00FF00",
        "status_message_font_size": 24,
        "use_online_messages": false,        
...        
\end{lstlisting}

The display of the text can be enabled using the 'use\_online\_messages' flag, the other parameters configure text size, text captions and text coloring.

\subsection{Geographic View Disable Rotate}

In the file 'compass.json' the following parameters can be set:

\begin{lstlisting}
...
        "disable_geographicview_rotate": false,
...        
\end{lstlisting}

\subsection{Auto-Resume From Live:Paused}
\label{sec:live_paused_resume}

In the file 'compass.json' the following values can be adapted to change how often the auto-resume question is asked, and the dialog wait time until the auto-resume is performed:

\begin{lstlisting}
...
        "auto_live_running_resume_ask_time": 60,
        "auto_live_running_resume_ask_wait_time": 1,
...        
\end{lstlisting}


