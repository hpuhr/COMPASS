\subsection{Layers Tab}

In the 'Layers' tab, a tree view is given to configure the display of the existing elements. 

\begin{figure}[H]
    \includegraphics[width=8cm,frame]{../screenshots/osgview_layers.png}
  \caption{OSG View layers}
\end{figure}

The following main tree elements are:\\

\begin{itemize}
 \item Geometry: Shows current DBO elements
 \item Measurements: Shows the current distance measurements
 \item Map: Shows current map layers
\end{itemize} 
 \ \\

In the Layers widget, a number of operations are possible for each tree item.

\begin{table}[H]
  \center
  \begin{tabular}{ | l | l | l |}
    \hline
    \textbf{Operation} & \textbf{Trigger} &  \textbf{Description} \\ \hline
    View sub-items & Triangle & Opens or closes view of the sub-items \\ \hline
    Display item & Checkbox & Enables or disables display of items (and all sub-items) \\ \hline
    Display context menu & Click on symbol & Opens the items context menu \\ \hline
  \end{tabular}
  \caption{Layer operations}
\end{table}

The context menu allows several actions to be performed on an item. If an item has sub-items, the same action will automatically be performed on the child items.

\subfile{osg_layers_geo_ops}

\subfile{osg_layers_measure_ops}

\subfile{osg_layers_map_ops}
