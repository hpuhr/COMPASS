\section{Opening a database}

To run the application, simply execute it. It is recommended to run the application from a console terminal (for issue analysis of the log text), but that is not mandatory.

\subsection{Configuration Upgrade}

If a previous version of ATSDB has been run on the workstation, the configuration (and additional application data) might be out of date. Since the framework of ATSDB is quite sensitive on being run with the latest configuration, the previous one will be overwritten, which will unfortunately clear any previous changes made by the user. \\\\

If an outdated configuration is detected, a dialog will be shown.

\begin{figure}[H]
    \includegraphics[width=12cm,frame]{../screenshots/config_data_update.png}
  \caption{Configuration and data update}
  \label{fig:db_connect}
\end{figure}

If 'Yes' is selected, the update is performed and the application starts. If 'No' is selected, the application will quit. If previous changes to an outdated configuration are if strong importance, please contact the author for support.

\subsection{Application Startup}
\label{sec:startup}

While ATSDB is more a database framework, it comes with a dedicated client. When the ATSDB client is started a dialog  for opening a database is shown. 

\begin{figure}[H]
  \hspace*{-2cm}
    \includegraphics[width=18cm,frame]{../screenshots/db_config_connect.png}
  \caption{Connecting to a database}
  \label{fig:db_connect}
\end{figure}

On the left-hand side, a database system can be selected.  Choices are either MySQL database or a file container with a SQLite3 database. \\
On the lower left (depending on the database system) either a MySQL server connection can be configured or a list of SQLite3 files is shown.\\

On the right-hand side a database schema can be selected and edited (editing is only recommended for experienced users).

\subsection{Connecting to a MySQL Server}

\begin{figure}[H]
  \center
    \includegraphics[width=7cm,frame]{../screenshots/mysql_server_selection.png}
  \caption{Connecting to a MySQL server}
  \label{fig:mysql_connect}
\end{figure}

Several MySQL servers can be defined, each one has a specific set of parameters. To add a new server, press the 'Add' button and enter a unique server name. To select the currently used server, use the dropdown menu. To delete the currently used server, press the 'Delete' button.

For connecting to a MySQL database, several parameters have to be entered:

\begin{table}[H]
  \center
  \begin{tabular}{ | l | l | l |}
    \hline
    \textbf{Parameter} & \textbf{Description} & \textbf{Example Values} \\ \hline
    Server Host & Network identifier of server & 'localhost', '10.0.0.123' \\ \hline
    Username & MySQL user name & 'sassc', 'root' \\ \hline
    Password & MySQL user password & 'sassc', '' \\ \hline
    Port Number & MySQL server port & '3306' \\
    \hline
  \end{tabular}
  \caption{MySQL server parameters}
\end{table}

To connect to a defined MySQL server, press the 'Connect' button.\\

\subsubsection{Access to SASS-C MySQL Servers}

Please note that it is not recommended to use SASS-C databases on which actual performance evaluations are to be performed. Using ATSDB, database operations can be performed which might impede results later obtained by using SASS-C. For this reason, it is recommended to either clone an evaluation or use one on which no later SASS-C evaluations are performed.

If a remote server is used, e.g. a SASS-C workstation, remote access might be prohibited, which will result in an access permissions error during connecting. To resolve this, (generally) remote access can be enabled in the servers MySQL configuration. As a superuser, edit the file 'my.cnf', which is commonly found under '/etc/mysql/my.cnf'. 

Find the line that states:
\begin{verbatim}
bind-address = 127.0.0.1
\end{verbatim}

Change the line to:

\begin{verbatim}
#bind-address = 127.0.0.1
\end{verbatim}

Then, restart the MySQL server using one the following commands:

\begin{verbatim}
/etc/init.d/mysqld restart

#OR, depending on your distribution
service mysql restart
\end{verbatim}

Then, to allow access to the databases, log into a MySQL client on the server as root and execute the following commands:

\begin{verbatim}
# log in as root, must be done as superuser
mysql -u root

# grant access rights for your MySQL user, 
# e.g. 'sass', from the your local IP address, 
# e.g. '192.168.0.104', using your password, e.g. 'sassc'
GRANT ALL ON *.* to 'sassc'@'192.168.0.104' IDENTIFIED BY 'sassc';

# set access rights
FLUSH PRIVILEGES;

# exit the MySQL client
exit
\end{verbatim}

After executing these steps once, remote access to this MySQL server from the specified IP address is enabled.

\paragraph{Firewall}

It might also be the case that your installation of SASS-C has an enabled firewall. If that is the case, access to be MySQL ports must be enabled.

\begin{figure}[H]
  \center
    \includegraphics[width=15cm,frame]{../screenshots/centos_firewall.png}
  \caption{MySQL Server firewall example}
\end{figure}

\subsubsection{Error Messages}

If  a  wrong  database  name  or  IP  address  is  used,  error  messages  can  be  e.g.  \\

\begin{figure}[H]
  \center
    \includegraphics[width=9cm,frame]{../screenshots/mysql_connect_error.png}
  \caption{MySQL Server not found error.}
\end{figure}

 or 

\begin{figure}[H]
  \center
    \includegraphics[width=9cm,frame]{../screenshots/mysql_user_error.png}
  \caption{MySQL user incorrect error.}
\end{figure}

If such an error occurs, correcting the server host name and or user/password should solve the problem.

\subsubsection{Opening a MySQL Database}

After successful connection, all existing databases in the server are shown in the 'Database name' drop-down menu. The last used one is selected automatically.

\begin{figure}[H]
  \center
    \includegraphics[width=5.5cm,frame]{../screenshots/mysql_database_selection.png}
  \caption{Selecting a MySQL database.}
  \label{fig:mysql_db_select}
\end{figure}

Several actions are available:

\begin{itemize}  
\item Drop-down selection: Selects the current database
\item New button: Allows creation of a new database
\item Clear button: Deletes all data with current database (after confirmation)
\item Delete button: Deletes current database (after confirmation)
\item Open button: Opens the current database
\end{itemize}

\subsubsection{Importing a MySQL Database}

After opening a database, import functions are available by clicking the 'Import' button:

\begin{figure}[H]
  \center
    \includegraphics[width=9cm,frame]{../screenshots/database_import.png}
  \caption{Importing a MySQL database}
\end{figure}

\paragraph{Importing a MySQL Text File}

A previously exported MySQL database can be read from a text file and written into the current database. After selecting this option, a file open dialog is shown which allows selection of '*.sql' files. Please note the following points:

\begin{itemize}  
\item The text file must contain valid MySQL statements
\item MySQL functions/views are not imported, only data which can be inspected
\item If more than 3 errors occur during the import process, it is aborted
\item If the import process was aborted, the current database might contain already imported parts which are not deleted automatically
\end{itemize}

\paragraph{Importing a MySQL Text Archive File}

A previously exported MySQL database can be read from a text archive file and written into the current database. After selecting this option, a file open dialog is shown which allows selection of '*.tar.gz *.gz *.tar *.zip *.tgz *.rar' files. Please note the following points:

\begin{itemize}  
\item This function does \textbf{not} work to import an Verif-exported \textit{evaluation tgz}, but imports an \textit{SQL archive} from inside such an evaluation tgz
\item The text archive file must follow the same points as in the \textbf{Importing a MySQL Text File} section.
\item All files within the archive are read and imported into the database
\item If more than 3 errors occur during the import process, it is aborted
\item If the import process was aborted, the current database might contain already imported parts which are not deleted automatically
\end{itemize}

\paragraph{Importing a SASS-C Evaluation Export}

The following steps must be taken:

\begin{itemize}  
\item An export from a SASS-C evaluation must exist, e.g. 'example.tgz'
\item Using your favorite archive manager, extract the file 'tgz-tmp/<JOB\_NAME>/exported.sql.gz'
\item After successfully connecting to a server, create a new database, e.g. '<job\_name>', and open it
\item Click on the 'Import' button and select 'Import MySQL Text Archive File'
\item Select the previously extracted 'exported.sql.gz'
\end{itemize}

\subsection{Opening a SQlite3 File Container}
\label{sec:sqlite_fc}
For opening a file container, three actions are available:

\begin{itemize}  
\item New: Creates a new (empty) SQLite3 container and adds it to the 'File Selection' list
\item Add: Adds an existing  SQLite3 container and adds it to the 'File Selection' list
\item Remove: Removes an entry from the 'File Selection' list
\end{itemize}

After selection of the wanted database container, the 'Open' button can be used for opening the database container.

\begin{figure}[H]
  \center
    \includegraphics[width=8cm,frame]{../screenshots/sqlite3_open.png}
  \caption{Opening a SQLite3 file container}
  \label{fig:sqlite3_open}
\end{figure}

\subsection{Database Schema Selection}
For a common user, usage of the pre-configured database schema 'SCDB' is recommened. To select a different database schema, please use the 'Schema selection' drop-down menu.\\

\begin{figure}[H]
  \center
    \includegraphics[width=8cm,frame]{../screenshots/database_schema_selection.png}
  \caption{Selecting a database schema}
  \label{fig:db_schema_select}
\end{figure}

For experienced users, a database schema can be created, edited and selected, which is currently not recommended (since it is not user friendly and might crash if used in the ``wrong'' manner).

\begin{figure}[H]
  \hspace*{-1cm}
    \includegraphics[width=16cm,frame]{../screenshots/database_schema_configuration.png}
  \caption{Configuring a database schema}
  \label{fig:db_schema_configuration}
\end{figure}

\subsection{DBO Management}

Also, after locking the database schema, the existing Database Objects can be edited, which is currently also not recommended, except for editing of data sources. There will be additional information about this subject in a later release, when the functionality has matured.

\subsubsection{Editing of Data Sources}
\label{sec:data_source_editing}

If a database generated by SASS-C is opened, the data sources are already defined. But, if data is imported from JSON, no data source information is given, so this information must either be edited manually, or loaded from a previous SASS-C evaluation. \\

For this purpose, data source information for each DBO can be stored in the configuration, and can be synchronized to/from a database. \\

To edit a data source, lock the schema and press the gear symbol {\includegraphics[scale=0.02]{../../data/icons/edit.png}.

\begin{figure}[H]
  \hspace*{-1.5cm}
    \includegraphics[width=18cm,frame]{../screenshots/dbo_edit.png}
  \caption{DBO Edit Widget}
  \label{fig:dbo_edit}
\end{figure}

Then, in the top-left corner, press the 'Edit Data Sources' button. If this is done for the first time with an empty database, the widget will look as follows:

\begin{figure}[H]
  \hspace*{-1cm}
    \includegraphics[width=16cm,frame]{../screenshots/dbo_edit_ds.png}
  \caption{DBO Edit Data Sources Widget}
  \label{fig:dbo_edit_ds}
\end{figure}

On the left hand side, the data sources in the configuration are shown and can be edited. A new one can be added using the 'Add New' button, and synchronization to database data sources can be proposed by using the 'Sync to DB' button. \\

In the middle, proposed synchronization are shown. They can be checked, edited or disabled before executing. \\

On the right hand side, the data sources as existing in the currently opened database are shown and can also be edited, as well as synchronized to the configuration. \\

The idea is that all commonly used data sources are defined and persisted in the configuration, and are then used automatically during the JSON import process. \\

Please \textbf{note} that currently the position of data sources is only required for \textbf{Radar} data sources (in the plot position calculation), for the others it would suffice to have SAC/SIC and name information for display purposes.

\paragraph{Editing Database Data Sources}

If SDDL data was imported without stored data source information in the configuration, the widget would look as follows:

\begin{figure}[H]
  \hspace*{-1cm}
    \includegraphics[width=16cm,frame]{../screenshots/dbo_edit_ds_db.png}
  \caption{DBO Edit Data Sources in Database}
  \label{fig:dbo_edit_ds_db}
\end{figure}

The orange fields represent the NULL value, to indicate that this information is not (yet) given. Each of the fields can edited, except for the ID field. \\

In a SASS-C database, all of the values would be set. The same can be achieved by editing. \\

\begin{figure}[H]
  \hspace*{-2cm}
    \includegraphics[width=18cm,frame]{../screenshots/dbo_edit_ds_db2.png}
  \caption{DBO Edit Data Sources in Database Edited}
  \label{fig:dbo_edit_ds_db2}
\end{figure}

\paragraph{Synchchronizing Database Data Sources to Config}

Now, with the already defined data sources, the values should be persisted in the configuration. To achieve this, click on the 'Sync to Config' button to create the proposed actions.

\begin{figure}[H]
  \hspace*{-2cm}
    \includegraphics[width=18cm,frame]{../screenshots/dbo_edit_ds_sync2cfg.png}
  \caption{DBO Edit Synchronize DB Data Sources to Configuration }
  \label{fig:dbo_edit_ds_sync2cfg}
\end{figure}

Since no previous data sources are exist in the configuration, the proposed action is to add all data sources as new ones. Unwanted actions can be disabled with the checkbox or set to 'None' in the drop-down menu. To perform the selected action, click the 'Perform Actions' button.

\begin{figure}[H]
  \hspace*{-2cm}
    \includegraphics[width=18cm,frame]{../screenshots/dbo_edit_ds_db2cfgsynced.png}
  \caption{DBO Edit Synchronized DB Data Sources to Configuration }
  \label{fig:dbo_edit_ds_db2cfgsynced}
\end{figure}

\paragraph{Editing Configuration Data Sources}

The configuration data sources displayed on the left can also be edited, to change names or correct information.

\begin{figure}[H]
  \hspace*{-2cm}
    \includegraphics[width=18cm,frame]{../screenshots/dbo_edit_ds_cfg.png}
  \caption{DBO Edit Data Sources in Configuration}
  \label{fig:dbo_edit_ds_cfg}
\end{figure}

A few values where changed, and the changes will be persisted on correct shutdown of the application.

\paragraph{Synchchronizing Configuration Data Sources to DB}

In the previous step, a few values were changed, and for the data source 'WAM1' the SAC/SIC values were also changed. To synchronize these changes to the currently opened database, click on the 'Sync to DB' button.

\begin{figure}[H]
  \hspace*{-2cm}
    \includegraphics[width=18cm,frame]{../screenshots/dbo_edit_ds_sync2db.png}
  \caption{DBO Edit Synchronize Configuration Data Sources to DB}
  \label{fig:dbo_edit_ds_sync2db}
\end{figure}

Note that the proposed action are to overwrite the data source information for all, except for 'WAM1', where no action is proposed since no matching SAC/SIC could be found. The perform the selected actions, press the 'Perform Actions' button.

\begin{figure}[H]
  \hspace*{-2cm}
    \includegraphics[width=18cm,frame]{../screenshots/dbo_edit_ds_cfg2dbsynced.png}
  \caption{DBO Edit Synchronized Configuration Data Sources to DB}
  \label{fig:dbo_edit_ds_cfg2dbsynced}
\end{figure}


\paragraph{Final Comments}

If the procedure seems complex, please remember that this procedure  has to be performed once for each DBO, after that the data sources are defined in the configuration. Also, if you have access to a SASS-C database, it is easier to synchronize the data source information from there to the configuration. \\

After doing so, for imported JSON the SAC/SIC information supplied in the ASTERIX target reports are used to match them to the respective data source. \\

For matching purposes only the SAC/SIC values are of importance, the name information doesn't have to be unique. If multiple data sources identical SAC/SIC values exist (for whatever reason), the first match is used during import.

\subsection{Tasks}
\label{sec:tasks}

\subsubsection{Importing ASTERIX Data}
\label{sec:asterix_import}

To execute this task select ''Task``->  ''Import ASTERIX Data` in the top menu bar. \\

Please note that the following framings are currently supported:
\begin{itemize}  
\item None: Raw, netto, unframed ASTERIX data blocks
\item IOSS: IOSS Final Format
\item RFF: Comsoft RFF format
\end{itemize}

Please note that the following categories and editions are currently supported:
\begin{itemize}  
\item 001
\begin{itemize}  
\item Edition 1.1
\end{itemize}
\item 002
\begin{itemize}  
\item Edition 1.0
\end{itemize}
\item 019
\begin{itemize}  
\item Edition 1.3
\end{itemize}
\item 020
\begin{itemize}  
\item Edition 1.5
\end{itemize}
\item 021
\begin{itemize}  
\item Edition 2.1
\end{itemize}
\item 023
\begin{itemize}  
\item Edition 1.2
\end{itemize}
\item 034
\begin{itemize}  
\item Edition 1.26
\end{itemize}
\item 048
\begin{itemize}  
\item Edition 1.15
\end{itemize}
\item 062
\begin{itemize}  
\item Edition 1.12
\end{itemize}
\item 063
\begin{itemize}  
\item Edition 1.0
\end{itemize}
\item 065
\begin{itemize}  
\item Edition 1.1
\end{itemize}
\end{itemize}


\begin{figure}[H]
  \center
    \includegraphics[width=14cm,frame]{../screenshots/asterix_import_data.png}
  \caption{Import ASTERIX Data task}
\end{figure}

In the 'File Selection' list, a list of available ASTERIX data files is given. Entries can be added using the 'Add' button or removed using the 'Remove' buttons. Please add and select the file to be imported.\\

In the 'ASTERIX Configuration' widget the framing, which categories are to be encoded and the currently acitive editions can be set.

Further below, the list of existing JSON object parsers is shown. Each parser is specialized for a specific DBO. To inspect or edit a JSON object parser, click on it in the list. This step is not required for parsing.

\paragraph{Importing}
Using the 'Test Import' button, the import function can be tested without inserting the data into the database, the 'Import' imports the selected file with the given options.

During import a status indication will be shown as follows:

\begin{figure}[H]
  \center
    \includegraphics[width=8cm,frame]{../screenshots/asterix_import_status.png}
  \caption{Import ASTERIX data task status}
\end{figure}

After import, a confirmation will be shown as follows:

\begin{figure}[H]
  \center
    \includegraphics[width=8cm,frame]{../screenshots/asterix_import_done.png}
  \caption{Import ASTERIX data task done}
\end{figure}

Please note that importing performance strongly depends on CPU performance (multi-threading very beneficial), but a import of 2.5 million target reports takes about 2 minutes on the author's hardware. \\

Please also note that an import can only be performed once, after that the application has to be closed and started again if additional ASTERIX importing is wanted. \\

Please also note the currently only the main data fields (as shown in the JSON object parsers) are imported, and that the timestamps of CAT001 are still only partial timestamps.


\subsubsection{Importing JSON Data}
\label{sec:json_import}

For information about test dataset please refer to \nameref{sec:test_data}. \\

To execute this task select ''Task``->  ''Import JSON Data` in the top menu bar.

\begin{figure}[H]
  \center
    \includegraphics[width=14cm,frame]{../screenshots/import_json_data.png}
  \caption{Import JSON Data task}
\end{figure}

In the 'File Selection' list, a list of available JSON data files is given. Entries can be added using the 'Add' button or removed using the 'Remove' buttons. Please note that either native JSON text files can be imported, or archives containing the same with the following formats: *.zip, *.gz, *.tgz \\

Below, the JSON parsing schema can be selected, the following options exist:
\begin{itemize}  
\item SDDL: Import JSON created by the SDDL v0.2 tool
\item ADSBexchange: Import JSON from the ADSBexchange platform
\item OpenSkyNetwork: Import JSON from the OpenSky Network platform
\end{itemize}

A new schema can be added using the ``Add'' button, or an existing removed using the ``Remove'' button.

Further below, for the selected parsing schema, the list of existing JSON object parsers is shown. Each parser is specialized for a specific DBO, in the shown example of OpenSkyNetwork, only ADSB exists, since no other data is provdided.

To inspect or edit a JSON object parser, click on it in the list. This step is not required for parsing.

\begin{figure}[H]
  \hspace*{-1cm}
    \includegraphics[width=16cm,frame]{../screenshots/import_json_data_object_parser.png}
  \caption{Import JSON Data Object Parser}
\end{figure}

If selected, on the right hand side shows a configuration widget for the selected parser.

\paragraph{JSON Object Parser Configuration}
At the top, a number of configuration options are given:

\begin{itemize}  
\item JSON Container Key: JSON key identifier where nested target reports are stored
\item JSON Key: JSON key identifier for selective parsing
\item JSON Value: JSON value for selective parsing
\item Override Data Source: Checkbox to specify if data source value should be overridden
\item Data Source Variable: Data source variable, for if data source value should be overridden
\end{itemize}


Below list of JSON to DBO variable mappings, with the following options:

\begin{itemize}  
\item Active: Checkbox if mapping should be used for import, mandatory ones must be active
\item JSON key: JSON key identifier
\item DBOVariable: DBO variable that data is mapped to
\item Mandatory: Sets if JSON key/value have to be present, otherwise the JSON object is skipped
\item Unit: Unit of the JSON value
\item Format: Parsing specialization for JSON value
\end{itemize}

A new mapping can be added using the ``Add'' button.

\paragraph{Importing}
Using the 'Test Import' button, the import function can be tested without inserting the data into the database, the 'Import' imports the selected file with the given options.

During import a status indication will be shown as follows:

\begin{figure}[H]
  \center
    \includegraphics[width=9cm,frame]{../screenshots/json_import_status.png}
  \caption{Import JSON data task status}
\end{figure}

If datasizes can be read from the archive (supported by ZIP but not by GZIP), predictions about the remaining time will be shown.

After import, a confirmation will be shown as follows:

\begin{figure}[H]
  \center
    \includegraphics[width=9cm,frame]{../screenshots/json_import_done.png}
  \caption{Import JSON data task done}
\end{figure}

Please note that importing performance strongly depends on CPU performance (multi-threading very beneficial), but a SDDL JSON import of 2.5 million target reports takes about 7 minutes on the author's hardware. \\

Please note that, since ADSBexchange JSON is not compliant to JSON standard (at least from the used test datasets), parsing errors will be shown and only a small percentage of target reports is imported. This will be improved in the near future.\\

Please also note that importing in the 'Simple' database schema will currently not work.  This will be improved in the near future.

\subsubsection{Calculate Radar Plot Position}

Please note that if no Radar target reports exist in the database, this steps does not have to be performed.\\

Since in SASS-C Verif radar plot coordinates are not given as latitude/longitude, which are the main coordinates for all processing in ATSDB, optionally these coordinates can be re-calculated and set in the database using the ''Calculate Radar Plot Position`` Task. \\

To execute this task select ''Task``->  ''Calculate Radar Plot Position` in the top menu bar. \\

Please note that for this step the data source positions have to set in the database, otherwise no plot position can be calculated. If a database generated by SASS-C is used, this should already be the case. For ones created by SDDL JSON, the steps described in \nameref{sec:data_source_editing} have to be performed once.

\begin{figure}[H]
  \center
    \includegraphics[width=14cm,frame]{../screenshots/task_calc_radar.png}
  \caption{Calculate radar plot position task}
  \label{fig:task_calc_radar}
\end{figure}

There are two projection methods (radar polar coordinates to WGS-84 coordinates) available. The \textit{RS2G} projection is the currently recommended option.

\paragraph{OGR Projection}

The EPSG code for the projection has to be chosen according to your needs, please refer to \url{http://spatialreference.org/ref/epsg/} for a list of possible codes.

The WGS84 latitude/longitude coordinates are then calculated using the radar positions in the database, the range and the azimuth. Please \textbf{note} that currently there will be offsets in the projected coordinates compared to the e.g. the ARTAS projection. The reason for this is under investigation.

\paragraph{RS2G Projection}

For this projection, no additional attributes must be given. Please \textbf{note} that while this projection is based on a common ``radar slant to geodesic transformation'', it is also not equivalent to e.g. the ARTAS projection. This will be improved in the near future, but validation was estimated to not to be taken lightly, and therefore the additional effort will be done before the next major release.

\paragraph{Calculation}

Press ``Calculate'' to start the calculation process, which will take a few minutes depending on the data size. \\

Messages like these will be printed in the text console, the last one indicates completion of the task:

The following windows will be shown, to give indication about the calculation progress:

\begin{figure}[H]
  \center
    \includegraphics[width=4cm,frame]{../screenshots/task_calc_radar_load.png}
  \caption{Calculate radar plot position task: Loading}
\end{figure}

\begin{figure}[H]
  \center
    \includegraphics[width=4cm,frame]{../screenshots/task_calc_radar_process.png}
  \caption{Calculate radar plot position task: Processing}
\end{figure}

If there were projection errors in the calculation, the before insertion the user will be asked to confirm. The coordinates with projection errors will of course not be committed to the database, but will be skipped.

\begin{figure}[H]
  \center
    \includegraphics[width=9cm,frame]{../screenshots/task_calc_radar_insert.png}
  \caption{Calculate radar plot position task: Insertion question on errors}
\end{figure}


\begin{figure}[H]
  \center
    \includegraphics[width=4cm,frame]{../screenshots/task_calc_radar_write.png}
  \caption{Calculate radar plot position task: Writing}
\end{figure}

\begin{figure}[H]
  \center
    \includegraphics[width=8cm,frame]{../screenshots/task_calc_radar_done.png}
  \caption{Calculate radar plot position task: Done}
\end{figure}

After running this task once (per database), the radar plots also have a set latitude/longitude. As stated, a post-processing step is recommended after executing this task. \\

Please \textbf{note} that this task can be re-run with different projections if wanted.

Please \textbf{note} that no additional Radar plot information is changed, only the latitude/longitude variables are updated.
 

\subsection{Starting}

After the previous steps have been completed, the 'Start' button can be pressed to continue. \\

\begin{figure}[H]
  \center
    \includegraphics[width=9cm,frame]{../screenshots/start.png}
  \caption{Starting}
\end{figure}

When a database is opened the first time, a post-processing will be performed automatically.

\subsubsection{Postprocessing}
When a database is imported, some information that eases usage of the software does not exist. This information is generated once during a post-processing step, which is automatically performed the first opening of a new database. If wanted, it can always performed using the 'Force post-processing' checkbox.

\begin{figure}[H]
  \hspace*{-2cm}
    \includegraphics[width=18cm,frame]{../screenshots/db_postprocessing.png}
  \caption{Post-processing a database}
  \label{fig:db_postprocessing}
\end{figure}

The following information is generated and stored in the database:

\begin{itemize}  
\item List of all active data sources for all DBOs
\item List with all minima/maxima for all variables of all DBOs
\end{itemize}

Please \textbf{note} that this step has to be performed only once for each database, and may take up to a few minutes for large datasets. \\

Please also \textbf{note} that during this step, no DBO data itself is changed, but only additional information is generated and stored in separate database tables.
 
