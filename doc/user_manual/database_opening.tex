\section{Opening a database}
\label{sec:startup}

While ATSDB is more a database framework, it comes with a dedicated client. When the ATSDB client is started a dialog  for opening a database is shown. 

\begin{figure}[H]
  \hspace*{-2cm}
    \includegraphics[width=18cm,frame]{../screenshots/db_config_connect.png}
  \caption{Connecting to a database}
  \label{fig:db_connect}
\end{figure}

On the left-hand side, a database system can be selected.  Choices are either MySQL database or a file container with a SQLite3 database. \\
On the lower left (depending on the database system) either a MySQL server connection can be configured or a list of SQLite3 files is shown.\\

On the right-hand side a database schema can be selected and edited (editing is only recommended for experienced users).

\subsection{Connecting to a MySQL Server}

\begin{figure}[H]
  \center
    \includegraphics[width=8cm,frame]{../screenshots/mysql_server_selection.png}
  \caption{Connecting to a MySQL server}
  \label{fig:mysql_connect}
\end{figure}

Several MySQL servers can be defined, each one has a specific set of parameters. To add a new server, press the 'Add' button and enter a unique server name. To select the currently used server, use the dropdown menu. To delete the currently used server, press the 'Delete' button.

For connecting to a MySQL database, several parameters have to be entered:

\begin{table}[H]
  \center
  \begin{tabular}{ | l | l | l |}
    \hline
    \textbf{Parameter} & \textbf{Description} & \textbf{Example Values} \\ \hline
    Server Host & Network identifier of server & 'localhost', '10.0.0.123' \\ \hline
    Username & MySQL user name & 'sassc', 'root' \\ \hline
    Password & MySQL user password & 'sassc', '' \\ \hline
    Port Number & MySQL server port & '3306', '' \\
    \hline
  \end{tabular}
  \caption{MySQL server parameters}
\end{table}

To connect to a defined MySQL server, press the 'Connect' button.\\

\subsubsection{Access to SASS-C MySQL Servers}

Please note that it is not recommended to use SASS-C databases on which actual performance evaluations are to be performed. Using ATSDB, database operations can be performed which might impede results later obtained by using SASS-C. For this reason, it is recommended to either clone an evaluation or use one on which no later SASS-C evaluations are performed.

If a remote server is used, e.g. a SASS-C workstation, remote access might be prohibited, which will result in an access permissions error during connecting. To resolve this, (generally) remote access can be enabled in the servers MySQL configuration. As a superuser, edit the file 'my.cnf', which is commonly found under '/etc/mysql/my.cnf'. 

Find the line that states:
\begin{verbatim}
bind-address = 127.0.0.1
\end{verbatim}

Change the line to:

\begin{verbatim}
#bind-address = 127.0.0.1
\end{verbatim}

Then, restart the MySQL server using one the following commands:

\begin{verbatim}
/etc/init.d/mysqld restart

#OR, depending on your distribution
service mysql restart
\end{verbatim}

Then, to allow access to the databases, log into a MySQL client on the server as root and execute the following commands:

\begin{verbatim}
# log in as root, must be done as superuser
mysql -u root

# grant access rights for your MySQL user, 
# e.g. 'sass', from the your local IP address, 
# e.g. '192.168.0.104', using your password, e.g. 'sassc'
GRANT ALL ON *.* to' 'sassc'@'192.168.0.104' IDENTIFIED BY 'sassc';

# set access rights
FLUSH PRIVILEGES;

# exit the MySQL client
exit
\end{verbatim}

After executing these steps once, remote access to this MySQL server from the specified IP address is enabled.

\subsubsection{Error Messages}

If  a  wrong  database  name  or  IP  address  is  used,  error  messages  can  be  e.g.  \\

\begin{figure}[H]
  \center
    \includegraphics[width=9cm,frame]{../screenshots/mysql_connect_error.png}
  \caption{MySQL Server not found error.}
\end{figure}

 or 

\begin{figure}[H]
  \center
    \includegraphics[width=9cm,frame]{../screenshots/mysql_user_error.png}
  \caption{MySQL user incorrect error.}
\end{figure}

If such an error occurs, correcting the server host name and or user/password should solve the problem.

\subsubsection{Opening a MySQL Database}

After successful connection, all existing databases in the server are shown in the 'Database name' drop-down menu. The last used one is selected automatically.

\begin{figure}[H]
  \center
    \includegraphics[width=5.5cm,frame]{../screenshots/mysql_database_selection.png}
  \caption{Selecting a MySQL database.}
  \label{fig:mysql_db_select}
\end{figure}

Several actions are available:

\begin{itemize}  
\item Drop-down selection: Selects the current database
\item New button: Allows creation of a new database
\item Clear button: Deletes all data with current database (after confirmation)
\item Delete button: Deletes current database (after confirmation)
\item Open button: Opens the current database
\end{itemize}

\subsubsection{Importing a MySQL Database}

After opening a database, import functions are available by clicking the 'Import' button:

\begin{figure}[H]
  \center
    \includegraphics[width=9cm,frame]{../screenshots/database_import.png}
  \caption{Importing a MySQL database}
\end{figure}

\paragraph{Importing a MySQL Text File}

A previously exported MySQL database can be read from a text file and written into the current database. After selecting this option, a file open dialog is shown which allows selection of '*.sql' files. Please note the following points:

\begin{itemize}  
\item The text file must contain valid MySQL statements
\item MySQL functions/views are not imported, only data which can be inspected
\item If more than 3 errors occur during the import process, it is aborted
\item If the import process was aborted, the current database might contain already imported parts which are not deleted automatically
\end{itemize}

\paragraph{Importing a MySQL Text Archive File}

A previously exported MySQL database can be read from a text archive file and written into the current database. After selecting this option, a file open dialog is shown which allows selection of '*.tar.gz *.gz *.tar *.zip *.tgz *.rar' files. Please note the following points:

\begin{itemize}  
\item The text file must follow the same points as in the \textbf{Importing a MySQL Text File} section.
\item All files within the archive are read and imported into the database
\item If more than 3 errors occur during the import process, it is aborted
\item If the import process was aborted, the current database might contain already imported parts which are not deleted automatically
\end{itemize}

\paragraph{Importing a SASS-C Evaluation Export}

The following steps must be taken:

\begin{itemize}  
\item An export from a SASS-C evaluation must exist, e.g. 'example.tgz'
\item Using your favorite archive manager, extract the file 'tgz-tmp/<JOB\_NAME>/exported.sql.gz'
\item After successfully connecting to a server, create a new database, e.g. '<job\_name>', and open it
\item Click on the 'Import' button and select 'Import MySQL Text Archive File'
\item Select the previously extracted 'exported.sql.gz'
\end{itemize}

\subsubsection{SQlite3 File container}
For opening a file container, clicking the 'Select' button opens a file selection dialog, in which any SQLite3 file can be selected as data source.

\begin{figure}[H]
  \center
    \includegraphics[width=8cm,frame]{../screenshots/sqlite3_open.png}
  \caption{Opening a SQLite3 file container}
  \label{fig:sqlite3_open}
\end{figure}

\subsection{Database Schema Selection}
For a common user, selection of a pre-configured database schema is recommened. To select a different database schema, please use the 'Schema selection' drop-down menu.\\

\begin{figure}[H]
  \center
    \includegraphics[width=8cm,frame]{../screenshots/database_schema_selection.png}
  \caption{Selecting a database schema}
  \label{fig:db_schema_select}
\end{figure}

For experienced users, on the right-hand side a database schema can be selected and edited, which is currently not recommended (since it is not user friendly and might crash if used in the ``wrong'' manner).

\begin{figure}[H]
  \hspace*{-1cm}
    \includegraphics[width=16cm,frame]{../screenshots/database_schema_configuration.png}
  \caption{Configuring a database schema}
  \label{fig:db_schema_configuration}
\end{figure}

\subsection{Starting}

After the previous steps have been completed, the 'Start' button can be pressed to continue. \\

When a database is opened the first time, a post-processing has to be performed.

\subsubsection{Postprocessing}
When a database is generated by a previous process,  some information that eases usage of the software does not exist. This information is generated once during a post-processing step, which is automatically performed. If wanted, it can always performed using the 'Force post-processing' checkbox.

Please \textbf{note} that during post-processing the application will not react to user input. \\


\begin{figure}[H]
  \hspace*{-2cm}
    \includegraphics[width=18cm,frame]{../screenshots/db_postprocessing.png}
  \caption{Post-processing a database}
  \label{fig:db_postprocessing}
\end{figure}

The following information is generated and stored in the database:

\begin{itemize}  
\item List of all active data sources for all DBOs
\item List with all minima/maxima for all variables of all DBOs
\end{itemize}

Please \textbf{note} that this step has to be performed only once for each database, and may take up to a few minutes for large datasets. \\

Please also \textbf{note} that during this step, no DBO data itself is changed, but only additional information is generated and stored in separate database tables.
 
