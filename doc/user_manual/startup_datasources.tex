\section{Manage Data Sources}
\label{sec:manage_datasources}

If a database generated by SASS-C is opened, the data sources are already defined. But, if data is imported from ASTERIX or JSON, data source information might be missing, so this information must either be edited manually, or loaded from a previous SASS-C evaluation. \\

For this purpose, data source information for each DBO can be stored in the configuration, and can be synchronized to/from a database. \\

To edit a data source, lock the schema and press the gear symbol {\includegraphics[scale=0.02]{../../data/icons/edit.png}.

\begin{figure}[H]
  \hspace*{-1.5cm}
    \includegraphics[width=18cm,frame]{../screenshots/dbo_edit.png}
  \caption{DBO Edit Widget}
  \label{fig:dbo_edit}
\end{figure}

Then, in the top-left corner, press the 'Edit Data Sources' button. If this is done for the first time with an empty database, the widget will look as follows:

\begin{figure}[H]
  \hspace*{-1cm}
    \includegraphics[width=16cm,frame]{../screenshots/dbo_edit_ds.png}
  \caption{DBO Edit Data Sources Widget}
  \label{fig:dbo_edit_ds}
\end{figure}

On the left hand side, the data sources in the configuration are shown and can be edited. A new one can be added using the 'Add New' button, and synchronization to database data sources can be proposed by using the 'Sync to DB' button. \\

In the middle, proposed synchronization are shown. They can be checked, edited or disabled before executing. \\

On the right hand side, the data sources as existing in the currently opened database are shown and can also be edited, as well as synchronized to the configuration. \\

The idea is that all commonly used data sources are defined and persisted in the configuration, and are then used automatically during the JSON import process. \\

Please \textbf{note} that currently the position of data sources is only required for \textbf{Radar} data sources (in the plot position calculation), for the others it would suffice to have SAC/SIC and name information for display purposes.

\subsection{Editing Database Data Sources}

If SDDL data was imported without stored data source information in the configuration, the widget would look as follows:

\begin{figure}[H]
  \hspace*{-1cm}
    \includegraphics[width=16cm,frame]{../screenshots/dbo_edit_ds_db.png}
  \caption{DBO Edit Data Sources in Database}
  \label{fig:dbo_edit_ds_db}
\end{figure}

The orange fields represent the NULL value, to indicate that this information is not (yet) given. Each of the fields can edited, except for the ID field. \\

In a SASS-C database, all of the values would be set. The same can be achieved by editing. \\

\begin{figure}[H]
  \hspace*{-2cm}
    \includegraphics[width=18cm,frame]{../screenshots/dbo_edit_ds_db2.png}
  \caption{DBO Edit Data Sources in Database Edited}
  \label{fig:dbo_edit_ds_db2}
\end{figure}

\subsection{Synchchronizing Database Data Sources to Config}

Now, with the already defined data sources, the values should be persisted in the configuration. To achieve this, click on the 'Sync to Config' button to create the proposed actions.

\begin{figure}[H]
  \hspace*{-2cm}
    \includegraphics[width=18cm,frame]{../screenshots/dbo_edit_ds_sync2cfg.png}
  \caption{DBO Edit Synchronize DB Data Sources to Configuration }
  \label{fig:dbo_edit_ds_sync2cfg}
\end{figure}

Since no previous data sources are exist in the configuration, the proposed action is to add all data sources as new ones. Unwanted actions can be disabled with the checkbox or set to 'None' in the drop-down menu. To perform the selected action, click the 'Perform Actions' button.

\begin{figure}[H]
  \hspace*{-2cm}
    \includegraphics[width=18cm,frame]{../screenshots/dbo_edit_ds_db2cfgsynced.png}
  \caption{DBO Edit Synchronized DB Data Sources to Configuration }
  \label{fig:dbo_edit_ds_db2cfgsynced}
\end{figure}

\subsection{Editing Configuration Data Sources}

The configuration data sources displayed on the left can also be edited, to change names or correct information.

\begin{figure}[H]
  \hspace*{-2cm}
    \includegraphics[width=18cm,frame]{../screenshots/dbo_edit_ds_cfg.png}
  \caption{DBO Edit Data Sources in Configuration}
  \label{fig:dbo_edit_ds_cfg}
\end{figure}

A few values where changed, and the changes will be persisted on correct shutdown of the application.

\subsection{Synchchronizing Configuration Data Sources to DB}

In the previous step, a few values were changed, and for the data source 'WAM1' the SAC/SIC values were also changed. To synchronize these changes to the currently opened database, click on the 'Sync to DB' button.

\begin{figure}[H]
  \hspace*{-2cm}
    \includegraphics[width=18cm,frame]{../screenshots/dbo_edit_ds_sync2db.png}
  \caption{DBO Edit Synchronize Configuration Data Sources to DB}
  \label{fig:dbo_edit_ds_sync2db}
\end{figure}

Note that the proposed action are to overwrite the data source information for all, except for 'WAM1', where no action is proposed since no matching SAC/SIC could be found. The perform the selected actions, press the 'Perform Actions' button.

\begin{figure}[H]
  \hspace*{-2cm}
    \includegraphics[width=18cm,frame]{../screenshots/dbo_edit_ds_cfg2dbsynced.png}
  \caption{DBO Edit Synchronized Configuration Data Sources to DB}
  \label{fig:dbo_edit_ds_cfg2dbsynced}
\end{figure}


\subsection{Final Comments}

If the procedure seems complex, please remember that this procedure  has to be performed once for each DBO, after that the data sources are defined in the configuration. Also, if you have access to a SASS-C database, it is easier to synchronize the data source information from there to the configuration. \\

After doing so, for imported JSON the SAC/SIC information supplied in the ASTERIX target reports are used to match them to the respective data source. \\

For matching purposes only the SAC/SIC values are of importance, the name information doesn't have to be unique. If multiple data sources identical SAC/SIC values exist (for whatever reason), the first match is used during import. 
