
\section{Reports}
\label{sec:reports}

\begin{figure}[H]
    \hspace*{-2.5cm}
    \includegraphics[width=19cm,frame]{figures/reports.png}
  \caption{Reports overview, showing an Evaluation report}
\end{figure}

In the 'Reports' tool, reports generated by certain tasks are listed for further inspection. 
In this section general concepts about reports are explained, information on a certain task's special type 
of report will be provided in the respective task's chapter later on. \\

When certain tasks are carried out (e.g. an Evaluation is ran), they will generate a new report, which is then persisted 
to the database and added to the list of reports. These reports can then be managed by using the controls in the top row
of the 'Reports' tool. \\

Please \textbf{note} that running a task might overwrite an existing report. \\

\subsection{Report Management}

A report can be selected in the combo box at the 'Report' tool's top at any time for further inspection. 
This combo box will subsequently be referred to as \textit{report selection}. 
Selecting a report will immediately show the report's main section. \\

Apart from selecting and visualizing a report, further actions can be carried out using the management buttons next to the 
report selection. These actions always apply to the currently selected report.

\begin{table}[H]
    \center
    \begin{tabular}{ | l | l | l |}
      \hline
      \textbf{Icon} & \textbf{Text} & \textbf{Description} \\ \hline
      \includegraphics[width=0.5cm,frame]{../../data/icons/refresh.png}/\includegraphics[width=0.5cm,frame]{../../data/icons/lock.png} & \nameref{sec:report_refresh} & Refresh the currently selected report (if needed) \\ \hline
      \includegraphics[width=0.5cm,frame]{../../data/icons/delete.png} & \nameref{sec:report_delete} & Remove the currently selected report \\ \hline
      \includegraphics[width=0.5cm,frame]{../../data/icons/save.png} & \nameref{sec:report_export} & Export the currently selected report to multiple formats \\ \hline
    \end{tabular}
    \caption{Report Management Actions}
\end{table}

\subsubsection{Refresh Report}
\label{sec:report_refresh}

When a task is run and a new report is generated, the report is in sync with the current state of the data in the database.
Modifying the database might result in a report to become outdated, in the sense that its input data or conditions have now changed.
For instance, recomputing the unique targets by running reference reconstruction will alter the data basis
for any Evaluation report generated before, as it relies in computed unique targets as input data. 
Further, as some reports are interactive and incorporate existing data residing in the database into their visualizations,
a report's features might become restricted on certain database changes. \\

Thus, each report obtains a state, which signals if it needs a (manual) refresh in order to once again become up-to-date with the current state of the database. 
This state is visualized by the icon of the 'Refresh' button next to the report selection. The report can be in one of the following three states. \\

\includegraphics[width=0.5cm,frame]{figures/refresh_uptodate.png} \textbf{Up-to-date}: The report is up-to-date, no refresh is needed. \\

\includegraphics[width=0.5cm,frame]{figures/refresh_needed.png} \textbf{Out-of-date}: The report is out-of-date, a manual refresh is needed. \\

\includegraphics[width=0.5cm,frame]{figures/refresh_lock.png} \textbf{Locked}: Report is in lock mode. 
When in lock mode, a report's features might be limited until the report is manually refreshed and thus unlocked. \\

Pressing the 'Refresh' button will update the report accordingly (possibly by running some recomputations) and it will again be set to up-to-date.
When in lock mode the report will be unlocked.

\subsubsection{Remove Report}
\label{sec:report_delete}

The user will be prompted for removal, and on accepting the report will be removed from the database.

\subsubsection{Export Report}
\label{sec:report_export}

A report can be exported by clicking the \includegraphics[width=0.5cm,frame]{../../data/icons/save.png} button.
A menu will open, giving multiple options for export.

\begin{figure}[H]
    \center
    \includegraphics[width=6cm,frame]{figures/export_options.png}
  %\caption{Data Sources Overview}
\end{figure}

\begin{itemize}
  \item \nameref{sec:report_export_json}: Export to a JSON file with additional linked resources
  \item \nameref{sec:report_export_latex}: Export to a Latex file with additional linked resources
  \item \nameref{sec:report_export_pdf}: Export to a PDF file
\end{itemize} 
\  \\

\subsubsection{Export as JSON}
\label{sec:report_export_json}

This export will generate a JSON file and a list of exported resources which are linked in the JSON file. \\

The following dialog will open.

\begin{figure}[H]
    \hspace*{-2.5cm}
    \center
    \includegraphics[width=12cm,frame]{figures/export_json.png}
  \caption{Export as JSON}
\end{figure}

\begin{itemize}  
    \item Base Directory: Directory the report files are exported to
    \item Report Directory: Report sub-directory created inside the specified base directory
    \item Report Name: Name of the report file created
    \item Author: Name of the report's author 
    \item Comments: Optional comments
\end{itemize}
\ \\

Please \textbf{note}: The specified report sub-directory will be created inside the specified base directory.
The content of the report directory will be completely removed in this process.
Never store important data in the specified report directory. \\

When pressing 'Export' the report will be exported using the specified settings.
After export the folder structure created will e.g. look like this.

\begin{figure}[H]
    \hspace*{-2.5cm}
    \center
    \includegraphics[width=10cm,frame]{figures/export_json_result.png}
  \caption{Export as JSON}
\end{figure}

The folder 'tables' contains all tabular data in JSON format linked in the main JSON file. \\
The folder 'screenshots' contains all image data linked in the main JSON file.

\subsubsection{Export as Latex}
\label{sec:report_export_latex}

\subsubsection{Export as PDF}
\label{sec:report_export_pdf}

\section{Reports}





\begin{itemize}  
    \item Figures: Viewable data, which can be visualized in certain Views and exported to a file as image data
    \item Tables: Tabular data, which can be inspected and exported
    \item Texts: Textual data
\end{itemize}


A report consists of sections, which are often linked to each other.
The user can step through these sections and inspect a section's contents.
At the moment three types of section contents exist.

\begin{itemize}  
    \item Figures: Viewable data, which can be visualized in certain Views and exported to a file as image data
    \item Tables: Tabular data, which can be inspected and exported
    \item Texts: Textual data
\end{itemize}
\ \\

