\section{Appendix: Latex Installation}
\label{sec:appendix_latex}  

To be able to use the report generation (View points as well as Evaluation reports), a Latex installation has to be installed on the used workstation. \\

Depending on what Linux operating system is used, the commands will differ. In this section installation instructions for a few selected OS' is provided.

\subsection {Ubuntu \& Debian Variants}

\begin{lstlisting}
sudo apt-get install texlive-full 
\end{lstlisting}

\subsection {CentOS \& Fedora Variants}

The following command should work:

\begin{lstlisting}
sudo yum install 'texlive-*'
\end{lstlisting}

There is still a guide which provides a more cumbersome installation - since the author does not use CentOS both options are not yet verified. For the more elaborate installtion please refer to \href{https://www.systutorials.com/how-to-install-tex-live-on-centos-7-linux/}{How To Install Tex Live on CentOS 7}.

\subsection{Testing}

After installtion, the following command should run an generate a similar output:

\begin{lstlisting}
$ pdflatex --version
pdfTeX 3.14159265-2.6-1.40.21 (TeX Live 2020/Debian)
kpathsea version 6.3.2
Copyright 2020 Han The Thanh (pdfTeX) et al.
There is NO warranty.  Redistribution of this software is
covered by the terms of both the pdfTeX copyright and
the Lesser GNU General Public License.
For more information about these matters, see the file
named COPYING and the pdfTeX source.
Primary author of pdfTeX: Han The Thanh (pdfTeX) et al.
Compiled with libpng 1.6.37; using libpng 1.6.37
Compiled with zlib 1.2.11; using zlib 1.2.11
Compiled with xpdf version 4.02
\end{lstlisting}

