\section{Display}

\subsection{Textual Display}

For textual analysis of the data, the Listbox View gives a highly configurable tabular text representation of the ATC surveillance data. \\

The following features currently exist:

\begin{itemize}  
\item Display of data as text tables
\item Configurable data loading of data of interest
\item Exporting of data as CSV
\end{itemize} 
\ \\

For further details please refer to \nameref{sec:listbox_view}.

\subsection{Graphical Display}
There also exists an integrated display solution, called the Geographic View (\textbf{O}pen\textbf{S}cene\textbf{G}raph View). It is not released as source code, and is included in the AppImage as binary component only. \\

The following features currently exist:

\begin{itemize}  
\item High-performance display based on OpenSceneGraph (e.g. 16 million target reports displayed after \textasciitilde60s)
\item Customizable map/terrain display based on osgEarth
\item Customizable display of ATC surveillance data
\item High-speed time-filtered display
\item Numerous operations for analysis, e.g. data selection, labeling, distance measurement
\item Configurable data layering and styling for detailed analysis
\item Relatively low memory footprint (e.g. 16 million target reports in \textasciitilde8 GB RAM)
\end{itemize} 
\ \\

For further details please refer to \nameref{sec:geo_view}.

\subsection{Cross-Selection}

For detailed analysis, functionality exists to inspect data of interest across all existing Views. For each View, what data is currently selected can be chosen and automatically updates all other Views. In each View, the currently selected data is given in a special representation. \\

Details on the data cross-selection features are given in each View.

\subsection{View Points}

View points are saved points of interest in the data, and can be used in a multitude of ways. A view point defines what data should be loaded (e.g. from what targets, in what time window), and includes a comment and state editable by the user.  \\
View points can either be created from inside the application (e.g. to highlight specific occurances) or imported from external applications (e.g. to make use of custom applications which find occurances of interest to the user). Using this feature can strongly improve efficiency of data inspection and assessment. \\

For further details please refer to \nameref{sec:ui_import_view_points}, \nameref{sec:view_points} and \nameref{sec:appendix_view_points}.
