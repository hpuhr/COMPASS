\chapter{Introduction}

The OpenATS COMPASS tool aims at providing a generalized framework for ATC surveillance data inspection and analysis. \\

It’s name being an abbreviation for \textbf{comp}liance \textbf{ass}essment, the OpenATS COMPASS tool allows air traffic surveillance recordings to be imported into a database for analysis, visualization and evaluation. \\

Many use-cases are supported, e.g. importing EUROCONTROL ASTERIX recordings into a database, textual \& visual analysis, evaluation (calculation of performance indicators) and evaluation report document generation. \\

The application is highly configurable and quite complex, and is developed for usage by air traffic surveillance professionals. To support new users, a user manual as well as YouTube videos are supplied. \\

The C++ code is released under the \href{https://www.gnu.org/licenses/gpl-3.0.en.html}{GPL-3.0}, while the Linux AppImage and the user manual are released under \href{https://creativecommons.org/licenses/by/4.0/}{CC BY 4.0}. \\

OpenATS COMPASS is publicly available and free for anyone to use (including commercial usage), under the previously stated licenses.

\section{User Manual}

This document has its focus on interaction and working procedures required to make use of the existing
functionality. In this introduction, feature highlights and acknowledgments are listed, followed by a brief summary of important aspects of  COMPASS in \nameref{sec:key_concepts}. \\

In the section \nameref{sec:installation}, prerequisites are listed and installation instructions are given. \\

In the larger section \nameref{sec:ui_overview}, the steps are described to run the application, create a database or access an existing one, import, process data, as well as load data into Views. \\

The filtering mechanism described in detail in \nameref{sec:filters}. \\

Based on Traffic of Opportunity reference trajectories can be calculated, the reconstructor feature is presented in the section \nameref{sec:reconstructor}. \\

The evaluation feature is presented in the section \nameref{sec:eval}, describing how requirement-based standards can be adapted/defined and compliance to said standards can be assessed. \\

Re-visiting saved points of interest is described in section \nameref{sec:view_points}. Inspection of loaded data using the existing Views is described in the sections \nameref{sec:table_view}, \nameref{sec:histo_view}, \nameref{sec:geo_view}, \nameref{sec:scatter_view} and \nameref{sec:grid_view}. \\

The application commonly uses the 'Offline' mode. When ASTERIX data is imported from the network, the application switches into 'Live' mode, which is described in \nameref{sec:live_mode}. \\

The automated execution of configured tasks is described in \nameref{sec:command_line}. \\

In the section \nameref{sec:troubleshooting} details about reported issues are collected. It also contains instructions on how to report new issues. \\

In the last section \nameref{sec:appendix} additional details are listed. 
The section \nameref{sec:appendix_utils} explains how data can be manually imported into COMPASS. 
In \nameref{sec:appendix_licensing} information is given about under which conditions COMPASS can be used, and what libraries with what licences are used in the background.

\pagebreak

\subfile{intro_feature_highlights}

\subfile{intro_general_aspects}

\subfile{intro_acknowledgements}

\subfile{intro_key_concepts}


