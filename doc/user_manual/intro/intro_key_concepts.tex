\section{Key Concepts}
\label{sec:key_concepts}

In this section, a few key concepts are introduced to convey a somewhat deeper understanding of COMPASS, and to allow the reader to understand some main design choices made by the author. 
This should also give indications about the strengths and drawbacks of the chosen approach.

\subsection*{Database Systems}
A database allows for storage, retrieval and filtering of the data of interest. While DuckDB - being a stand-alone database - also has some drawbacks, 
it was chosen for its performance and ease of use (compared to e.g. NoSQL databases or MySQL variants). \\\\
DuckDB encapsulates a database in a single file container, which is read from a storage medium (e.g. hard drive). 
Therefore, it does not require the installation of a database service, which is one of the advantages of the current solution. \\

\begin{itemize}
\item Single file container
\begin{itemize}
\item Can easily be copied, shared, archived, ..
\end{itemize}
\item DuckDB database
\begin{itemize}
\item Can be read/edited with other tools
\begin{itemize}
\item e.g. dbeaver, from Python scripts, ...
\end{itemize}
\item Nothing has to be additionally installed
\end{itemize}
\item High-performance, light-weight
\end{itemize}

\subsection*{Configuration}
At startup numerous configuration files are loaded, and at shutdown the current configuration state of COMPASS is saved.\\\\

The configuration is not just a matter of storing simple parameters of components, but also what components exist. To give an example: Each existing View is saved, and when the program is started again, the previously active Views are restored. The same is true for almost all components of COMPASS. \\\\

Using this configuration, a user can have a specific program configuration for a specific usage situation, which can be instantly reused for a different dataset, using a specific View or filter configuration, allowing for a high degree of flexibility and supporting numerous use-cases. \\


\begin{itemize}
\item Stored in a home folder for current AppImage version
\begin{itemize}
\item e.g. \textasciitilde/.compass/0.9.0
\end{itemize}
\item Read at application startup
\item Written at application shutdown (if wanted)
\item Not written at application crash
\item If not existing for current version
\begin{itemize}
\item Default configuration copied by AppImage
\end{itemize}
\end{itemize}

\subsection*{Data Sources \& Database Content}

Data sources are (at the most basic level) defined by a name and SAC/SIC, and obtain a certain data source type (\textbf{DSType}). 
If e.g. the data source 'Wuerzburg' of DSType 'Radar' is defined in the application, during an ASTERIX import the respective ASTERIX CAT048 data is inserted into the database and associated to the data source by SAC/SIC. \\

For each data source up to 4 different \textbf{lines} can be used (L1, L2, L3, L4). 
Such lines can either be used to distinguish data recorded from different network lines, but also to distinguish different recordings of the same data source. 
For instance, different tracker runs can be imported and analyzed by importing them into different data source lines. \\

A certain Database Content (\textbf{DBContent}) is defined by a name and obtains a collection of variables. 
For example, CAT048 and CAT062 are types of DBContent and each has variables holding time, position, Mode 3/A code and Mode C height, and so on. 
If such a DBContent is present, target reports can be loaded from a database and displayed.\\

\begin{itemize}
\item Data Source: Single data source, e.g. a certain Radar
\begin{itemize}
\item Identified by name (\textbf{DS ID}), SAC/SIC, ...
\end{itemize}
\item Data Source Type (\textbf{DSType})
\begin{itemize}
\item Type of source technology, i.e.
\begin{itemize}
\item Radar, MLAT, ADS-B, Tracker, RefTraj, Other
\end{itemize}
\end{itemize}
\item Line Identifier (\textbf{Line ID})
\begin{itemize}
\item L1, L2, L3, L4
\end{itemize}
\item Database Content (\textbf{DBContent}) 
\begin{itemize}
\item Type of data, e.g. ASTERIX Formats CAT048, CAT020, CAT021, ...
\item During import, DBContent is attributed to 
\begin{itemize}
\item Data source, Line ID
\end{itemize}
\item Contains DBContent Variables: Time of Day, Latitude, Longitude, ...
\end{itemize}
\end{itemize}

\subsection*{ASTERIX Data Import}
If surveillance data is given in EUROCONTROL's ASTERIX format, it can be decoded using the \textbf{jASTERIX} library. This library allows adding new framings, categories and editions based on configuration only. \\\\

The resulting JSON data is then mapped to DBContent variables stored in the database. The mapping between JSON keys and DBContent variables is configurable, allowing for a broad usage spectrum.

\subsection*{JSON Data Import}
If surveillance data is given in the JSON format, it can be mapped to DBContent variables and imported to a selected schema. Currently, only jASTERIX JSON is supported for importing. The mapping between JSON keys and DBContent variables is configurable (as with the ASTERIX Data Import), allowing for a broad usage spectrum.

\subsection*{Meta Variables}

To allow displaying data from different DBContent in the same system, so-called \textbf{Meta variables} were introduced, which hold variables that are present in some or all DBContent (with a possibly different name or unit). \\
For example, there is the meta variable 'Time of Day', which is a collection of sub-variables for each existing DBContent and the respective 'Time of Day' variable. \\

\begin{itemize}
\item Each DBContent has DBContent variables
\item The same variables exist in multiple DBContents
\item Meta variables
\begin{itemize}
\item Group DBContent variables of same content
\item Allow easier usage
\item Can be inspected using the Configuration menu
\end{itemize}
\end{itemize}

\subsection*{Unique Target Numbers}
A \textbf{U}nique \textbf{T}arget \textbf{N}umber (\textbf{UTN}) groups together DBContent target reports.
This information can be created either by a general association task or created based on ARTAS TRI information (one UTN for each ARTAS track, with the associated sensor information). \\

\begin{itemize}
\item Calculated during 'Calculate Associations' task
\item Each unique target is attributed a unique target number (\textbf{UTN})
\item Each target report can have
\begin{itemize}
\item No UTN: Not associated
\item A single UTN: Associated to 1 target
\item Multiple UTNs: Special case e.g. for merged PSR plots
\end{itemize}
\item Targets in the evaluation are identified by unique UTN
\item UTN can group together multiple tracks / flight legs
\item Allows comparison of data
\begin{itemize}
\item Find equivalent reference data for test data
\end{itemize}
\end{itemize}

\subsection*{Data Loading}
In COMPASS, a unified data loading process was chosen, meaning that only exactly one common dataset is loaded, which can be inspected using multiple Views. \\

When started, data is incrementally read from the database, stored in the resulting dataset, and distributed to the active Views. Each time such a loading process is triggered, all Views clear their dataset and gradually update. \\

This makes working with the data somewhat easier to understand, since only one dataset exists, while on the other hand it does not allow several independent datasets (e.g. with different filters) to be loaded at the same time. \\

\begin{itemize}
\item If a loading process is started, a dataset will be loaded into memory (RAM)
\begin{itemize}
\item DBContent as defined by data sources, filters
\item Only contains DBContent variables that are needed at the time
\end{itemize}
\item Only one dataset can exist at a time
\item Dataset is distributed and displayed in all views
\item If additional DBContent variables are needed
\begin{itemize}
\item e.g. by a View
\item a reload is required
\end{itemize}
\end{itemize}

\subsection*{Live Mode}
Commonly, COMPASS is used in the 'Offline' mode, which allows inspection of larger amounts of recorded ASTERIX data. \\

To analyze live data from the network, a 'Live: Running' mode exists which records data from UDP network streams, decodes the content, imports the data to the database, and immediately displays the data in the Geographic View. \\

The most current data (up to 5 minutes) is stored in main memory (RAM), to inspect cases of interest in the immediate past. \\

The 'Live: Running' can be paused into the 'Live: Paused' mode, which allows offline-like inspection, after which the mode can be resumed into the 'Live: Running' mode without loss of ASTERIX data. \\

The 'Live' modes can be quit, after which the application returns to the 'Offline' mode, to allow inspection of the full database as always.

\subsection*{Calculation of Reference Trajectories}

After importing ASTERIX data, reference trajectories can be calculated from the given Traffic of Opportunity. Based on loaded time-slices, target reports of the same target are associated to a UTN, which in turn is used to calculate the reference trajectories. \\

The main use case of calculated references is as reference data in evaluations. Special care should be taken that the reference accuracy needed for the respective evaluation is given. \\

\subsection*{Free / Pro License}

OpenATS COMPASS is publicly available and free for anyone to use (including commercial usage), under the \href{https://www.gnu.org/licenses/gpl-3.0.en.html}{GPL-3.0}, and \href{https://creativecommons.org/licenses/by/4.0/}{CC BY 4.0} licenses. \\

However, one component, namely an advanced reference trajectory calculation feature (Probabilistic + IMM Reconstructor) is only available under a commercial professional license. In the free license, a basic reconstructor (Scoring + UMKalman) is available.

\subsection*{Evaluation}

Based on reference and test data source(s), evaluation of configurable standards within defined airspace sectors can be performed. The results are given as sector averages as well as per-target statistics and can be inspected down to the per-target-report level if wanted. Certain targets can be removed from the evaluation (i.e. target not of interest, transponder issues), and evaluation report documents can be automatically generated as PDFs.

\subsection*{View Presets}

Each view can be configured to the user's needs, to suit various use-cases. For each type of view, a number of predefined (default) presets are delivered, which can be used to conveniently set a number of configuration options for specific display purposes. Existing presets can be changed and overridden, also new ones can be created.

\subsection*{View Points}
\label{sec:key_concepts_viewpoints}

A View Point can be seen as a conglomerate of filter settings, view configurations and displayable view annotations, 
which represents a unique view onto either internal data (COMPASS database) or externally provided data. \\

Key features of View Points include: \\

\begin{itemize}
    \item \textbf{Data Sources}: Specification for enabling/disabling certain data sources
    \item \textbf{Filters}: Filter statements for which data to load from the specified data sources
    \item \textbf{View Configurations}: Specification for regions of interest, time frames of interest, etc.
    \item \textbf{Annotations}: View annotations to display in COMPASS' various views (scatter data, histograms, grid data, geometric data, etc.) 
\end{itemize}

These key features reflect the data flow in COMPASS:  

\textit{Database} $\rightarrow$ \textit{Data Sources} $\rightarrow$ \textit{Filters} $\rightarrow$ \textit{Views} \\

Usage of each feature is optional, so a viewpoint may contain e.g. only a set of filter statements or only a set of annotations. \\

View Points can be used in various ways in COMPASS: \\

\begin{itemize}
    \item View Points can be exported from or imported into COMPASS using the 'View Points' tab in the main window (see \ref{sec:view_points}).
    This way a specific view onto data can be saved and restored at a later point.
    \item View Points also provide a simple and practical interface to provide external data to COMPASS via view annotations.
    \item They are also used heavily by COMPASS itself, e.g. to distribute evaluation results to its views.
\end{itemize}
\ \\

A View Point is stored in the widely-used JSON format and uses predefined tags, which are explained in more detail in 
the appendix (\ref{sec:appendix_view_points}).
