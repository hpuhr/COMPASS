\subsubsection{Map Operations}
\label{ref:geoview_map_ops}

To access the map operations, click the globe symbol \includegraphics[width=0.5cm,frame]{../../data/icons/globe.png}. Please note that only the main Map item has a context menu, and only allows setting of map files and changing of the opacity.

\begin{itemize}
 \item Map File: Change the background map
 \item Opacity: Change the background map opacity
\end{itemize} 

\paragraph{Changing the Background Map}

Please note that, while the default background map is supplied COMPASS, the other background map types are downloaded from public Internet sources and therefore require an Internet connection. They are then cached locally to facilitate faster access. \\

For each map layer defined in the background map, a checkbox is shown to disable the layer, and by clicking on the map layer symbol it's opacity can be changed.

To change the background map, click the globe symbol \includegraphics[width=0.5cm,frame]{../../data/icons/globe.png} in root Map layer to access the map selection. \\

The following maps are commonly available:

\begin{itemize}
 \item arcgis.earth*
 \item minimal.earth*
 \item openstreetmap.earth*
 \item openstreetmap\_dark.earth*
 \item openstreetmap\_german.earth*
 \item readymap.earth
 \item readymap-detailed.earth
\end{itemize}
 \ \\
 
 Please note that for each map marked with \textbf{*} a 3D version (as listed) and a 2D version (with filename suffix '\_2d') exists. Each of them contains similar content, but changes the display mode to 3D/2D upon selection. \\

 \textbf{Note}: If a map is selected which only contains either a 2D or 3D version, 
 the 'Switch Map Dimensions' button \includegraphics[width=0.5cm,frame]{../../data/icons/2d.png}/\includegraphics[width=0.5cm,frame]{../../data/icons/3d.png} in the toolbar will be deactivated.\\
 
The map loading and display is based on the geoEarth library (\url{http://geoearth.org/}), as are the map file definitions.  \\

The map which can be set using this dialog is simply a file list from the folder '\textasciitilde/.compass/data/maps'. So, changes can be made to the supplied ones or custom user maps can be added to this folder. \\
Please refer to section \nameref{sec:appendix_maps} for further details.

\newpage
\paragraph{ArcGIS Map}

As supplied in the geoEarth example files, this map data is obtained from ArcGIS Online (\url{https://doc.arcgis.com/en/arcgis-online/reference/what-is-agol.htm}). It shows satellite imagery, supplied with elevation data from ReadyMap. 

\begin{figure}[H]
    \hspace*{-2.5cm}
    \includegraphics[width=19cm,frame]{figures/geoview_arcgis.png}
  \caption{Geographic View Arcgis map}
\end{figure}

\newpage
\paragraph{Minimal Map}

This minimal map shows national borders based on an ESRI shapefile, provided by Bjorn Sandvik on \url{thematicmapping.org} and European airports, provided by \url{https://ec.europa.eu/eurostat/web/gisco/geodata/reference-data/transport-networks}.

\begin{figure}[H]
    \hspace*{-2.5cm}
    \includegraphics[width=19cm,frame]{figures/geoview_minimal.png}
  \caption{Geographic View minimal map}
\end{figure}

\newpage
\paragraph{Open Street Map}

This very useful map shows map data from \url{https://www.openstreetmap.org/}.

\begin{figure}[H]
    \hspace*{-2.5cm}
    \includegraphics[width=19cm,frame]{figures/geoview_osm.png}
  \caption{Geographic View OpenStreetMap}
\end{figure}

It is possible to zoom in to a very high level of detail, to even inspect airport layouts.

\begin{figure}[H]
    \hspace*{-2.5cm}
    \includegraphics[width=19cm,frame]{figures/geoview_osm_vienna.png}
  \caption{Geographic View OpenStreetMap Vienna Airport}
\end{figure}

\newpage
\paragraph{Open Street Map Dark}
\label{ref:geoview_map_osm_dark}

Additionally, there is a dark version available:
\begin{figure}[H]
    \hspace*{-2.5cm}
    \includegraphics[width=19cm,frame]{figures/geoview_osm_dark.png}
  \caption{Geographic View OpenStreetMap Dark}
\end{figure}

It is possible to zoom in to a very high level of detail, to even inspect airport layouts.

\begin{figure}[H]
    \hspace*{-2.5cm}
    \includegraphics[width=19cm,frame]{figures/geoview_osm_vienna_dark.png}
  \caption{Geographic View OpenStreetMap Vienna Airport Dark}
\end{figure}

\newpage
\paragraph{Open Street Map German}

This very useful map shows map data from \url{https://www.openstreetmap.de/}.

\begin{figure}[H]
    \hspace*{-2.5cm}
    \includegraphics[width=19cm,frame]{figures/geoview_osm_de.png}
  \caption{Geographic View OpenStreetMap German}
\end{figure}

It is possible to zoom in to a very high level of detail, to even inspect airport layouts.

\newpage
\paragraph{ReadyMap \& ReadyMap Detailed}

This map also shows satellite data, from \url{http://web.pelicanmapping.com/readymap-tiles/}.

\begin{figure}[H]
    \hspace*{-2.5cm}
    \includegraphics[width=19cm,frame]{figures/geoview_ready.png}
  \caption{Geographic View ReadyMap}
\end{figure}

This detailed version shows the same data as ReadyMap, but to a higher detail level. \\

Please note that this map includes an elevation layer, so mountains are modeled in 3D.

\begin{figure}[H]
    \hspace*{-2.5cm}
    \includegraphics[width=19cm]{figures/geoview_readymap_elav.png}
  \caption{Geographic View ReadyMap detailed elevation}
\end{figure}


