\section{Command Line Options}
\label{sec:command_line} 

Several command line options have been added to allow for semi-automated running of tasks or convenient usage. This also allows for (limited) automated batch processing of data. \\

\includegraphics[width=0.5cm]{../../data/icons/hint.png} Please note that configuration of the application still has to be performed using the GUI, therefore it is required to set up the application correctly before command line options can be used successfully.\\

\includegraphics[width=0.5cm]{../../data/icons/hint.png} Please also note that error or warning messages (or related confirmations) will still halt the automatic running of tasks, to ensure that the user is always aware of occurring issues. \\

To get a list of available command line options use e.g.:
\begin{lstlisting}
./COMPASS-release_x86_64.AppImage --help
\end{lstlisting}

\begin{lstlisting}
Allowed options:
  --help                                produce help message
  -r [ --reset ]                        reset user configuration and data
  --override_cfg_path arg               overrides 'default' config subfolder to other value, e.g. 'org'
  --expert_mode                         set expert mode
  --create_db arg                       creates and opens new DuckDB database  with given filename, e.g. '/data/file1.db'
  --open_db arg                         opens existing DuckDB database with given filename, e.g. '/data/file1.db'
  --import_data_sources_file arg        imports data sources JSON file with given filename, e.g. '/data/ds1.json'
  --import_view_points arg              imports view points JSON file with given filename, e.g. '/data/file1.json'
  --import_asterix_file arg             imports ASTERIX file with given filename, e.g. '/data/file1.ff'
  --import_asterix_files arg            imports multiple ASTERIX files with given filenames, e.g. '/data/file1.ff;/data/file2.ff'
  --import_asterix_pcap_file arg        imports ASTERIX PCAP file with given filename, e.g. '/data/file1.pcap'
  --import_asterix_pcap_files arg       imports multiple ASTERIX PCAP files with given filenames, e.g. '/data/file1.pcap;/data/file2.pcap'
  --import_asterix_file_line arg        imports ASTERIX file with given line, e.g. 'L2'
  --import_asterix_date arg             imports ASTERIX file with given date, in YYYY-MM-DD format e.g. '2020-04-20'
  --import_asterix_file_time_offset arg used time offset during ASTERIX file import Time of Day override, in  HH:MM:SS.ZZZ'
  --import_asterix_ignore_time_jumps    imports ASTERIX file ignoring 24h time jumps
  --import_asterix_network              imports ASTERIX from defined network UDP streams
  --import_asterix_network_time_offset arg
                                        additive time offset during ASTERIXnetwork import, in HH:MM:SS.ZZZ'
  --import_asterix_network_max_lines arg
                                        maximum number of lines per data source during ASTERIX network import, 1..4
  --import_asterix_network_ignore_future_ts 
                                        ignore future timestamps during ASTERIX network import'
  --asterix_framing arg                 sets ASTERIX framing, e.g. 'none', 'ioss', 'ioss_seq', 'rff'
  --asterix_decoder_cfg arg             sets ASTERIX decoder config using JSON string, e.g. ''{"10":{"edition":"0.31"}}'' (including one pair of single quotes)
  --import_asterix_parameters arg       ASTERIX import parameters as JSON string, e.g. ''{"filter_modec_active": true,"filter_modec_max": 50000.0,"filter_modec_min": -10000.0}'' (including one pair of single quotes)
  --import_json arg                     imports JSON file with given filename, e.g. '/data/file1.json'
  --import_gps_trail arg                imports gps trail NMEA with given filename, e.g. '/data/file2.txt'
  --import_gps_parameters arg           import GPS parameters as JSON string, e.g. ''{"callsign": "ENTRPRSE", "ds_name": "GPS Trail", "ds_sac": 0, "ds_sic": 0, "mode_3a_code": 961, "set_callsign": true, "set_mode_3a_code": true, "set_target_address": true, "target_address": 16702992,"tod_offset": 0.0}'' (including one pair of single quotes)
  --import_sectors_json arg             imports exported sectors JSON with given filename, e.g. '/data/sectors.json'
  --calculate_radar_plot_positions      calculate radar plot positions
  --calculate_artas_tr_usage            associate target reports based on ARTAS usage
  --reconstruct_references              reconstruct references from sensor and tracker data
  --load_data                           load data after start
  --export_view_points_report arg       export view points report after start with given filename, e.g. '/data/db2/report.tex
  --evaluate                            run evaluation
  --evaluation_parameters arg           evaluation parameters as JSON string, e.g. ''{"current_standard": "test", "dbcontent_name_ref": "CAT062", "dbcontent_name_tst": "CAT020"}'' (including one pair of single quotes)
  --evaluate_run_filter                 run evaluation filter before evaluation
  --export_eval_report arg              export evaluation report after start with given filename, e.g. '/data/eval_db2/report.tex
  --max_fps arg                         maximum fps for display in GeographicView'
  --no_cfg_save                         do not save configuration upon quitting
  --open_rt_cmd_port                    open runtime command port (default at 27960)
  --enable_event_log                    collect warnings and errors in the event log
  --quit                                quit after finishing all previous steps
\end{lstlisting}
\ \\

If additional command line options are wanted please contact the author.

\subsection{Options}

\subsubsection{--create\_db filename}

Adds the supplied filename and creates a new DuckDB database.
 
\subsubsection{--open\_db filename}

Adds the supplied filename and opens an existing DuckDB database.

\subsubsection{--import\_data\_sources\_file filename}

Adds the data sources defined in the JSON file to the configuration, as described in \nameref{sec:config_ds_export}.

\subsubsection{--import\_view\_points filename}

After a database was opened, adds the supplied filename and starts an import using the task described in \nameref{sec:ui_import_viewpoints}.
 
\subsubsection{--import\_asterix\_file filename}
\subsubsection{--import\_asterix\_files filenames}
\subsubsection{--import\_asterix\_pcap\_file filename}
\subsubsection{--import\_asterix\_pcap\_files filenames}

After a database was opened, adds the supplied filename and starts an import using the task described in \nameref{sec:ui_import_asterix}.

\subsubsection{--import\_asterix\_file\_line arg}

If an import using the task described in \nameref{sec:ui_import_asterix} is started, it will use the given line identifier (L1 ...L4).


\subsubsection{--import\_asterix\_date arg}

If an import using the task described in \nameref{sec:ui_import_asterix} is started, it will use the import date in YYYY-mm-DD format.

\subsubsection{--import\_asterix\_network}

After a database was opened, an import using the task described in \nameref{sec:ui_import_asterix_network} is started.

\subsubsection{--import\_asterix\_network\_time\_offset arg}

If an import using the task described in \nameref{sec:ui_import_asterix_network} is started, it will use the given time offset, in HH:MM:SSS.

\subsubsection{--import\_asterix\_network\_max\_lines arg}

If an import using the task described in \nameref{sec:ui_import_asterix_network} is started, it will use the given maximum number of input lines (and deactivate the others).

\subsubsection{--asterix\_framing framing}

When an Import ASTERIX Task is started the given framing is used, the following options exist:

\begin{itemize}
\item none:  Raw, netto, unframed ASTERIX data blocks, equivalent to the 'empty' value in the GUI
\item ioss:  IOSS Final Format
\item ioss\_seq: IOSS Final Format with sequence numbers
\item rff: Comsoft RFF format
\end{itemize}

\subsubsection{--asterix\_decoder\_cfg 'str'}

When an Import ASTERIX task is started the given configuration is used, in which the editions and mapping can be specified for each category using a JSON string. \\

Using the following string the edition 0.31 can be set for category 010:  \\
'\{"10":\{"edition":"0.31"\}\}' (including one pair of single quotes) \\

In a nice formatting the string looks like this:
\begin{lstlisting}[basicstyle=\small\ttfamily]
'{
"10":
    {
        "edition":"0.31"
    }
}'
\end{lstlisting}
\ \\

Please note the string "10" to identify category 010. \\

For one or a number of categories, the following options can be set:

\begin{itemize}
\item "edition":  ASTERIX editions as string, e.g. "1.0"
\item "ref\_edition":  ASTERIX reserved expansion field as string, e.g. "1.9"
\item "spf\_edition": ASTERIX special purpose field as string, e.g. "ARTAS"
\item "mapping": Mapping used after decoding, e.g. "CAT010 to Radar"
\end{itemize}
\ \\

Please note that the naming must be exactly as in the GUI, otherwise the application quits with an error message.

\subsubsection{--import\_asterix\_parameters}

When an Import ASTERIX task is started the given configuration overrides are used, all parameters can be set using a JSON string. \\

In a nice formatting the string looks like this:
\begin{lstlisting}[basicstyle=\small\ttfamily]
'{
{
  "filter_position_active": true,
  "filter_latitude_min": 49.686,
  "filter_latitude_max": 50.3898,
  "filter_longitude_min": 8.0115,
  "filter_longitude_max": 9.1129,
  "filter_modec_active": true,
  "filter_modec_max": 500,
  "filter_modec_min": -10000
}'
\end{lstlisting}
\ \\

Please note that the parameter naming must be exactly as in the 'task\_import\_asterix.json', with the following additional non-configuration parameters:
\begin{lstlisting}[basicstyle=\small\ttfamily]
'{
  "override_tod_active": true,
  "filter_tod_active": true,
  "filter_position_active": true,
  "filter_modec_active": true,
  "file_line_id": 0, // 0...3
  "date_str": "YYYY-mm-DD",
  "network_ignore_future_ts": true,
  "obfuscate_secondary_info": true
}'
\end{lstlisting}
\ \\


\subsubsection{--import\_json\_file filename}

After a database was opened, adds the supplied filename and starts an import using the task described in \nameref{sec:ui_import_json}.

\subsubsection{--import\_gps\_trail filename}

After a database was opened, adds the supplied filename and starts an import using using the task described in \nameref{sec:ui_import_gps}.

\subsubsection{--import\_gps\_parameters 'str'}

When an Import GPS Trail task is started the given configuration is used, in which the data source and secondary parameters for the GPS trail is defined using a JSON string.

In a nice formatting the string looks like this:
\begin{lstlisting}[basicstyle=\small\ttfamily]
'{
  "callsign":"ENTRPRSE",
  "ds_name":"GPS Trail",
  "ds_sac":0,
  "ds_sic":0,
  "mode_3a_code":961,
  "set_callsign":true,
  "set_mode_3a_code":true,
  "set_target_address":true,
  "target_address":16702992,
  "tod_offset":0.0
}'
\end{lstlisting}
\ \\

Please \textbf{note} that both 'mode\_3a\_code' and 'target\_address' must be given as decimal values.

% \subsubsection{--import\_sectors\_json}
% 
% After a database was opened, adds imports a previously exported sectors JSON file using the following tasks:
% 
% \begin{itemize}
%  \item \nameref{sec:task_manage_sectors}
% \end{itemize}


\subsubsection{--import\_sectors\_json  filename}

After a database was opened, adds the sectors defined in supplied filename using the task described in \nameref{sec:ui_configure_sectors_manage}.

\subsubsection{--calculate\_artas\_tr\_usage}

After a database with imported content exists, target report association can be calculated based on ARTAS tracks using the task described in \nameref{sec:ui_associate_tr_artas}.

\subsubsection{--reconstruct\_references}

After a database with imported content exists, reconstruct reference trajectories using the task described in \nameref{sec:reconst}.

\subsubsection{--load\_data}

Triggers a load process after opening the database.

\subsubsection{--export\_view\_points\_report}

After starting the application into the management window, a export View Points as PDF process is triggered as described in
\nameref{sec:view_points_export_pdf}.

The given argument filename defines the report filename, with the report directory as the parent directory of the given filename.

\subsubsection{--evaluate}

After opening a database a pre-configured evaluation run is triggered as described in \nameref{sec:eval}.

\subsubsection{--import\_gps\_parameters 'str'}

When an Evaluation task is started the given configuration is used, in which all parameters of the evaluation can be defined using a JSON string.

In a nice formatting such a string can look like this:
\begin{lstlisting}[basicstyle=\small\ttfamily]
'{
  "active_sources_ref":{
    "CAT062":{
      "1234":true
    }
  },
  "active_sources_tst":{
    "CAT021":{
      "2345":true
    }
  },
  "current_standard":"Dubious Targets",
  "dbcontent_name_ref":"CAT062",
  "dbcontent_name_tst":"CAT021",
  "use_grp_in_sector":{
    "Dubious Targets":{
      "SectorName1":{
        "Optional":false
      },
      "SectorName2":{
        "Optional":false
      },
      "SectorName3":{
        "Optional":true
      },
      "SectorName4":{
        "Optional":false
      }
    }
  }
}'
\end{lstlisting}
\ \\

The configuration of such parameters is rather complex, and it is recommended to contact the author if detailed information is needed.

\subsubsection{--evaluate\_run\_filter}

When an Evaluation task is started an automatic filtering of targets is performed, as configured using the 'Filter UTNs' dialog defined in \nameref{sec:eval_filter_targets}.


\subsubsection{--export\_eval\_report}

After a pre-configured evaluation run is was performed a report PDF is generated as described in \nameref{sec:eval}.

The given argument filename defines the report filename, with the report directory as the parent directory of the given filename.

\subsubsection{--no\_cfg\_save}

When quitting the application later, no configuration changes are saved.

\subsubsection{--max\_fps arg}

Uses the given integer as maximum frames-per-second number in Geographic View, allows lessening the workload for slower workstations or when using remote-access. Default is 30, for increased performance 10 or 15 or even 0 are suitable values. With a value 0, the Geographic View switches to so-called on-demand rendering, rendering only frames with changed data content, resulting in the lowest performance requirements.

\subsubsection{--quit}

After the other tasks were run, automatically quits the application.

