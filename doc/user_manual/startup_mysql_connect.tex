\section{Connecting to a MySQL Server}
\label{sec:mysql_connect}

\begin{figure}[H]
  \center
    \includegraphics[width=7cm,frame]{../screenshots/mysql_server_selection.png}
  \caption{Connecting to a MySQL server}
  \label{fig:mysql_connect}
\end{figure}

Several MySQL servers can be defined, each one has a specific set of parameters. To add a new server, press the 'Add' button and enter a unique server name. To select the currently used server, use the dropdown menu. To delete the currently used server, press the 'Delete' button.

For connecting to a MySQL server, several parameters have to be entered:

\begin{table}[H]
  \center
  \begin{tabular}{ | l | l | l |}
    \hline
    \textbf{Parameter} & \textbf{Description} & \textbf{Example Values} \\ \hline
    Server Host & Network identifier of server & 'localhost', '10.0.0.123' \\ \hline
    Username & MySQL user name & 'sassc', 'root' \\ \hline
    Password & MySQL user password & 'sassc', '' \\ \hline
    Port Number & MySQL server port & '3306' \\
    \hline
  \end{tabular}
  \caption{MySQL server parameters}
\end{table}

To connect to a defined MySQL server, press the 'Connect' button.\\

\subsection{Access to SASS-C MySQL Servers}

Please note that it is not recommended to use SASS-C databases on which actual performance evaluations are to be performed. \\
Using ATSDB, database operations can be performed which might impede results later obtained by using SASS-C. For this reason, it is recommended to either clone an evaluation or use one on which no later SASS-C evaluations are performed.

If a remote server is used, e.g. a SASS-C workstation, remote access might be prohibited, which will result in an access permissions error during connecting. To resolve this, (generally) remote access can be enabled in the servers MySQL configuration. As a superuser, edit the file 'my.cnf', which is commonly found under '/etc/mysql/my.cnf'. 

Find the line that states:
\begin{cverbatim}
bind-address = 127.0.0.1
\end{cverbatim}

Change the line to:

\begin{cverbatim}
#bind-address = 127.0.0.1
\end{cverbatim}

Then, restart the MySQL server using one the following commands:

\begin{cverbatim}
/etc/init.d/mysqld restart

#OR, depending on your distribution
service mysql restart
\end{cverbatim}

Then, to allow access to the databases, log into a MySQL client on the server as root and execute the following commands:

\begin{cverbatim}
# log in as root, must be done as superuser
mysql -u root

# grant access rights for your MySQL user, 
# e.g. 'sass', from the your local IP address, 
# e.g. '192.168.0.104', using your password, e.g. 'sassc'
GRANT ALL ON *.* to 'sassc'@'192.168.0.104' IDENTIFIED BY 'sassc';

# set access rights
FLUSH PRIVILEGES;

# exit the MySQL client
exit
\end{cverbatim}

After executing these steps once, remote access to this MySQL server from the specified IP address is enabled.

\subsubsection{Firewall}

It might also be the case that your installation of SASS-C has an enabled firewall. If that is the case, access to be MySQL ports must be enabled.

\begin{figure}[H]
  \center
    \includegraphics[width=15cm,frame]{../screenshots/centos_firewall.png}
  \caption{MySQL Server firewall example}
\end{figure}

\subsection{Error Messages}

If a wrong database name or IP address is used, error messages can be e.g. \\

\begin{figure}[H]
  \center
    \includegraphics[width=9cm,frame]{../screenshots/mysql_connect_error.png}
  \caption{MySQL Server not found error.}
\end{figure}

 or 

\begin{figure}[H]
  \center
    \includegraphics[width=9cm,frame]{../screenshots/mysql_user_error.png}
  \caption{MySQL user incorrect error.}
\end{figure}

If such an error occurs, correcting the server host name and or user/password should solve the problem. 
