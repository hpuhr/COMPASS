\subsection{Import GPS Trails}
\label{sec:ui_import_gps}

This task allows importing (D)GPS trails into the opened database, by parsing an NMEA file and writing the position updates into the 'RefTraj' DBContent.

\begin{figure}[H]
  \center
    \includegraphics[width=16cm]{figures/gps_import_task.png}
  \caption{Import GPS Trail}
\end{figure}

There exist 2 tabs:

\begin{itemize}
\item Main: File path and information text
\item Config: Configuration of data source and secondary information
\end{itemize}
\ \\

\subsubsection{Main Tab}

At the top, a label exists indicating the file to be imported. \\

Below, a text field is given, which after selection of an NMEA file displays the content information and/or error messages. The following information is (commonly) presented:

\begin{itemize}
\item Number of lines parsed
\item Number of (position) fixes
\item Number of skipped fixes, skipped if
\begin{itemize}
\item the time  and position is the same as the previous update or
\item the quality is set to 0 (invalid position).
\end{itemize}
\item Number of remaining fixes (to be imported)
\item Number of missing speed vector information
\item Timestamps
\begin{itemize}
\item In NMEA (if certain messages are present) the date information is given
\item The Begin/End timestamps are the first and the last in the file
\item If this information is wrong, it is recommended to either edit the NMEA file, or (if no 24h rollover occurred) to use the 'Override Date' function
\end{itemize}
\end{itemize}

\subsubsection{Config Tab}

\begin{figure}[H]
    \includegraphics[width=16cm]{figures/gps_import_config.png}
  \caption{Import GPS Trail Config}
\end{figure}

In the configuration tab, several configuration parameters can be set:

\begin{itemize}
\item SAC: System area code of the reference data source
\item SIC: System identification code of the reference data source
\item Name: Name of the reference data source
\item Time of Day Offset: Time correction factor, in seconds. Set to 0 to disable.
\item Override Date: Set date information, e.g. if missing in NMEA file
\item Mode 3/A Code (octal): Mode 3/A code to be set. Uncheck checkbox to disable.
\item Target Address (hexadecimal): Mode S address to be set. Uncheck checkbox to disable.
\item Callsign: Target identification to be set. Uncheck checkbox to disable.
\item Line ID: Line to be used during import
\end{itemize}
\ \\

\subsubsection{Running}

After selecting an NMEA file, the task can be performed using the 'Import' button.


