
\subsection{Reconstruct References}
\label{sec:ui_proc_reconst_references}

This task allows the computation of reference trajectories from associated tracks with assigned unique target numbers (see \nameref{sec:ui_associate_tr}). \\

Currently the following sensor data can be incorporated into reference track estimation:

\begin{itemize}
    \item System Tracks
    \item ADS-B \\
\end{itemize}

It is possible to exclude target reports from the estimation based on various criteria. 
The used reconstructor then integrates positions, velocities and any associated uncertainties of 
added target reports, in order to calculate reference positions with estimated locations, velocities and uncertainties. \\

After reference computation the generated reference trajectories for each target will be added to the database, 
using a new data source. \\

The task's dialog will show as follows. \\

\begin{figure}[H]
    \center
      \includegraphics[width=14cm]{figures/ui_task_references_dialog.png}
    \caption{Calculate References Task - Configuration Dialog}
\end{figure}

It consists of several configuration tabs:

\begin{itemize}
    \item \textbf{Input Data Sources}: Configuration of data sources to use for reference computation.
    \item \textbf{Position Data Filter}: Filtering options for excluding certain target reports from reference computation.
    \item \textbf{Kalman Settings}: Configuration of the Kalman reconstructor.
    \item \textbf{Output}: Configuration of the generated data source. \\
\end{itemize}

These configuration tabs will be described in more detail below.

\subsubsection{Input Data Sources}

\begin{figure}[H]
    \center
      \includegraphics[frame,width=14cm]{figures/ui_task_references_tab_inputds.png}
    \caption{Calculate References Task - Input Data Sources Tab}
\end{figure}

In this dialog the data sources which are incorporated into reference trajectory estimation can be configured.
There exists a checkbox for each possible DSType and then a nested list of checkboxes for individual data sources. \\

\textbf{Note}: At least one data source has to be selected in order to run reference computation.

\subsubsection{Position Data Filter}

\begin{figure}[H]
    \center
      \includegraphics[frame,width=14cm]{figures/ui_task_references_tab_filter.png}
    \caption{Calculate References Task - Position Data Filter Tab}
\end{figure}

In this dialog various filters can be enabled in order to remove certain target reports from reference computation. 
The following tables describe the existing options.

\begin{table}[H]
    \center
    \begin{tabularx}{\textwidth}{ | l | l | X |}
        \hline
        \textbf{Parameter} & \textbf{Default} & \textbf{Description} \\ \hline
        Filter Position Data & enabled & Enables/disables filtering of position data \\ \hline
    \end{tabularx}
\end{table}

\textbf{Note}: All other options will be ignored if this option is disabled. \\

\textit{Tracker Position Data Usage}:
\begin{table}[H]
    \center
    \begin{tabularx}{\textwidth}{ | l | l | X |}
        \hline
        \textbf{Parameter} & \textbf{Default} & \textbf{Description} \\ \hline
        Only Use Confirmed & enabled & Use only non-tentative track updates \\ \hline
        Only Use Non-Coasting & enabled & Use only non-coasting track updates \\ \hline
        Only Use Detected Report & disabled & Use only non-zero Detection Type track updates \\ \hline
        Only Use Non-Single PSR-only Detections & disabled & Use no mono PSR-only track updates \\ \hline
        Only Use High Accuracy & enabled & Use only track updates with Std.Dev. smaller than threshold \\ \hline
        Minimum Position Stddev [m] & 30 & Std.Dev. threshold \\ \hline
    \end{tabularx}
\end{table}

\textit{ADS-B Position Data Usage}:
\begin{table}[H]
    \center
    \begin{tabularx}{\textwidth}{ | l | l | X |}
        \hline
        \textbf{Parameter} & \textbf{Default} & \textbf{Description} \\ \hline
        Only Use MOPS V1 / V2 & enabled & Use only MOPS V1 and V2 ADS-B data \\ \hline
        Only Use High NUCp / NIC & disabled & Use only ADS-B data larger than NUCp / NIC threshold \\ \hline
        Minimum Position NUCp / NIC & 4 &  NUCp / NIC threshold \\ \hline
        Only Use High NACp & enabled & Use only ADS-B data larger than NACp threshold \\ \hline
        Minimum Position NACp & 4 & NACp threshold \\ \hline
        Only Use High SIL & disabled & Use only ADS-B data larger than SIL threshold \\ \hline
        Minimum Position SIL & 1 & SIL threshold \\ \hline
    \end{tabularx}
\end{table}

\subsubsection*{Kalman Settings}

\begin{figure}[H]
    \center
      \includegraphics[frame,width=14cm]{figures/ui_task_references_tab_kalman.png}
    \caption{Calculate References Task - Kalman Settings Tab}
\end{figure}

In this dialog the used reference trajectory reconstructor can be chosen and configured.
The following tables describe the existing options.

\begin{table}[H]
    \center
    \begin{tabularx}{\textwidth}{ | l | l | X |}
        \hline
        \textbf{Parameter} & \textbf{Default} & \textbf{Description} \\ \hline
        Reconstructor & UMKalman2D & Reconstructor type used for reference computation \\ \hline
    \end{tabularx}
\end{table}

The currently implemented reconstructors are based on the well-known Kalman filter and its various variants.
Currently the following 'flavours' are implemented. \\

\begin{itemize}
    \item \textbf{Uniform Motion Kalman (UMKalman2D)}: 
        Classic Kalman filter assuming a constant velocity between consecutive measurements
    \item \textbf{Accelerated Motion Kalman (AMKalman2D)}: 
        Unscented Kalman filter assuming constant acceleration between consecutive measurements \\
\end{itemize}

\textit{Default Uncertainties}: Parameters related to the uncertainties used in the Kalman filter, in addition to those provided 
by the input data.
\begin{table}[H]
    \center
    \begin{tabularx}{\textwidth}{ | l | l | l | X |}
        \hline
        \textbf{Parameter} & \textbf{Default} & \textbf{Unit} & \textbf{Description} \\ \hline
        %Measurement Stddev & 30 & m & Default standard deviation for added measurements \\ \hline
        %Measurement Stddev (high) & 1000 & m & Default high standard deviation for added measurements, 
        %    used if important data is not provided by the data base (e.g. velocity) \\ \hline
        Process Stddev & 30 & m & Process noise standard deviation of the modelled Kalman process \\ \hline
        Tracker Velocity Stddev & 50 & m & Default standard deviation for tracker velocities, 
            used if not provided by the data \\ \hline
        %Tracker Acceleration Stddev & 50 & m & Default standard deviation for tracker accelerations,
        %    used if not provided by the data \\ \hline
        ADS-B Velocity Stddev & 50 & m & Default standard deviation for ADS-B velocities, 
            used if not provided by the data \\ \hline
        %ADS-B Acceleration Stddev & 50 & m & Default standard deviation for ADS-B accelerations, 
        %    used if not provided by the data \\ \hline
    \end{tabularx}
\end{table}

\textbf{Note}: When using the default settings, target reports without position accuracy data are not used for reference position estimation. However, if such filters (in the \textit{Position Data Filter} tab) are disabled, if no position accuracy is provided by the data for a specific target report, a standard value of 100 meters is automatically assumed. \\

Please also note that if an important value such as velocity is not provided by the data, a value of zero with a very high standard deviation of 1000 meters is automatically assumed. \\

\textit{Chain Generation}: Parameters related to criteria causing the reconstructor to reinitialize the Kalman filter, 
resulting in the reference trajectory of a track being split into multiple sub-chains.
\begin{table}[H]
    \center
    \begin{tabularx}{\textwidth}{ | l | l | l | X |}
        \hline
        \textbf{Parameter} & \textbf{Default} & \textbf{Unit} & \textbf{Description} \\ \hline
        Minimum Time Step & 0 & s & Minimum time difference between consecutive measurements (measurements obtaining a 
            lower time difference to their predecessor will be skipped) \\ \hline
        Maximum Time Step & 11 & s & Maximum time difference between consecutive measurements (measurements obtaining a 
            higher time difference to their predecessor will cause the Kalman filter to reset, 
            resulting in a new sub-chain) \\ \hline
        Minimum Chain Size & 2 & & Minimum number of created reference positions in a sub-chain, in order to add the chain 
            to the final result \\ \hline
    \end{tabularx}
\end{table}

\textit{Additional Options}:
\begin{table}[H]
    \center
    \begin{tabularx}{\textwidth}{ | X | l | l | X |}
        \hline
        \textbf{Parameter} & \textbf{Default} & \textbf{Unit} & \textbf{Description} \\ \hline
        Smooth Results & enabled & & Smooth the results of the Kalman filter using a Rauch–Tung–Striebel smoother \\ \hline
        Resample System Tracks & enabled & & Resample input system tracks using cubic spline interpolation \\ \hline
        Resample System Tracks - Resample Interval & 1 & s & Time interval at which system tracks are resampled \\ \hline
        Resample System Tracks - Maximum Time Step & 30 & s & Maximum time difference between consecutive tracker 
            reports in order to resample the segment \\ \hline
        Resample Result & enabled & & Resample the generated reference trajectories by interpolating the final Kalman states
            at a fixed rate \\ \hline
        Resample Result - Resample Interval & 2 & s & Time interval at which the result trajectories are resampled \\ \hline
        Resample Result - Resample Process Stddev & 10 & m & Standard deviation of the Kalman process when interpolating the 
            Kalman states during result resampling \\ \hline
    \end{tabularx}
\end{table}

Resampling of system tracks can be enabled in order to obtain similar sample rates for tracker and ADS-B
data, which reduces a sensor-specific bias of the Kalman filter. Cubic B-splines are used to interpolate the tracker target
reports in a relaxed way. In case a spline segment deviates too strongly from its interval, a linear interpolation is 
used instead. \\

Result resampling can be enabled in order to obtain reference trajectories with a homogeneous sample rate. 
It is carried out by interpolating the Kalman states at fixed time-steps. The accuracy of the resulting samples
follows the Kalman accuracies more closely when interpolating near the endpoints of a segment, and it will decrease
in the middle of a segment. The amount of decrease of accuracy will depend on the time duration of the segment and the used 
process standard deviation (parameter \textit{Resample Process Stddev}).

\subsubsection{Output}

\begin{figure}[H]
    \center
      \includegraphics[frame,width=14cm]{figures/ui_task_references_tab_info.png}
    \caption{Calculate References Task - Output Tab}
\end{figure}

Here the data source created by reference computation can be configured. 
Reference data will be stored to the database as part of this data source.
The following options are available. \\

\textit{Output Data Source}:
\begin{table}[H]
    \center
    \begin{tabularx}{\textwidth}{ | l | l | X |}
        \hline
        \textbf{Parameter} & \textbf{Default} & \textbf{Description} \\ \hline
        Name & CalcRefTraj & Name of the created data source \\ \hline
        SAC & 0 & SAC of the created data source \\ \hline
        SIC & 1 & SIC of the created data source \\ \hline
        Line ID & 1 & Line ID under which the created data is stored \\ \hline
    \end{tabularx}
\end{table}

Please \textbf{note} that an already existing reference data source is just updated by running the 
reference computation again. This way it is possible to store several reference trajectories to 
different line IDs for the sake of comparison.

\subsubsection{Running The Task}

If configured correctly, the task can be started by clicking the \textit{Run} button.
The current configuration can always be stored without running the task by clicking the \textit{Close} button. 
This can be useful when testing the configuration for using \nameref{sec:ui_proc_calc_references_preview}. \\

When the task is started, the following progress dialog is shown to the user.

\begin{figure}[H]
    \center
      \includegraphics[width=9cm]{figures/ui_task_references_dialog_result.png}
    \caption{Calculate References Task - Calculate References Status Dialog}
\end{figure}

First the elapsed time and the current status of the reconstruction process are shown.
Then a few sections give more inside into what happened in the various stages of the process, 
which are described in more detail below.

\paragraph{Loaded Data} This section gives insight into to what ratio data has been loaded from
the database. It shows total and currently loaded data counts, and loading percentage, broken down into
individual DSTypes.

\paragraph{Used Position Data} This section gives insight into to what ratio loaded data has been
included in the reconstruction process after position filtering. 
It shows total data counts, and the amount of data items used for reconstruction in absolute numbers and percentages,
broken down into individual DSTypes.

\paragraph{References Info} This section displays information about the reference computation result.
It shows the number of reference trajectories created, the total number of created target reports, and 
accuracy statistics, which can be used to quantify how well the reconstructor performed.

After finishing, the progress dialog may also display error information. Reference computation can mainly fail in two ways. \\

Badly configured data sources or non-optimal filter settings may result in no data going into the reference computation pipeline.
This will result in the error message \textit{'No input data for reference computation'}. In this case it may help to 
check the included data sources and the chosen filter settings in the reference computation configuration (see above). \\

In case the reconstructor fails to generate any reference data for the included data, the error 
\textit{'No reference data created'} will be shown. A non-optimal Kalman configuration might be the cause for this.

\subsubsection{Reconstruction Preview}
\label{sec:ui_proc_calc_references_preview}

After selecting a desired set of parameters, it can be tiresome to run reference computation for the entire 
set of targets in order to obtain a feedback on the quality of the result. In order to get a faster feedback on the 
chosen parameter set, a \textit{Reconstruction Preview} can be executed for a single chosen track. 
Unlike reference computation for all tracks, the computed reference trajectory will not be written to the database.
Instead it will be visualized in an open Geographic View as an annotation, along with additional intermediate data.
This intermediate data can be especially useful to inspect the various stages of the Kalman process. \\

For this feature, first a set of parameters can be selected in the \textit{Calculate References} dialog, as described above.
Then the current settings are stored by closing the dialog again using the \textit{Close} button on the bottom left. \\

Next, a desired Target can be selected in the \textit{Targets} tab of the main window (see \nameref{sec:ui_targets}).
The reconstruction preview can be started by right-clicking on the selected target and choosing \textit{Reconstruction Preview}.

\begin{figure}[H]
    \center
      \includegraphics[frame,width=16cm]{figures/ui_task_references_recprev_start.png}
    \caption{Starting a Reconstruction Preview} 
\end{figure}

After reconstruction has finished, the results can be inspected in the \textit{Annotations} node of Geographic View's \textit{Layer} tab.

\begin{figure}[H]
    \center
      \includegraphics[frame,width=16cm]{figures/ui_task_references_recprev_anno.png}
    \caption{Reconstruction Preview - Annotations} 
\end{figure}

Up to three trajectories are shown in different colors, each of them obtaining positions, speed vectors and error ellipses,
and each of them representing different stages of the reconstruction process.

\paragraph{Kalman} Shows the estimates coming directly out of the Kalman filter.
Using this annotation the performance of the Kalman filter can be visually inspected.

\begin{figure}[H]
    \center
      \includegraphics[width=10cm]{figures/ui_task_references_recprev_kalman.png}
    \caption{Reconstruction Preview - Annotation \textit{Kalman}} 
\end{figure}

\paragraph{Kalman (smoothed)} Shows the result of the RTS smoother (if smoothing enabled).
Using this annotation the performance of the RTS smoother can be visually inspected.

\begin{figure}[H]
    \center
      \includegraphics[width=10cm]{figures/ui_task_references_recprev_smoothed.png}
    \caption{Reconstruction Preview - Annotation \textit{Kalman (smoothed)}} 
\end{figure}

\paragraph{Kalman (resampled)} Shows the homogeneously resampled Kalman states (if result resampling enabled).
Using this annotation result resampling can be visually inspected.

\begin{figure}[H]
    \center
      \includegraphics[width=10cm]{figures/ui_task_references_recprev_final.png}
    \caption{Reconstruction Preview - Annotation \textit{Kalman (resampled)}} 
\end{figure}

Additionally, an annotation called \textit{Interpolated Tracks} shows the spline-resampled system tracks in grey color,
if system track resampling is enabled.

\begin{figure}[H]
    \center
      \includegraphics[width=10cm]{figures/ui_task_references_recprev_interp.png}
    \caption{Reconstruction Preview - Annotation \textit{Interpolated Tracks}} 
\end{figure}
