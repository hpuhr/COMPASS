\section{Use Cases}

Depending on the use case, different options exist. Please follow the listed references in your respective use case. Optional steps are marked with an asterisk \textbf{*}.

\subsection{Using a Verif Database}

\paragraph{Acessing an Existing Verif Database}

\begin{itemize}
 \item \nameref{sec:mysql_connect}
 \item \nameref{sec:mysql_open_db}
 \item \nameref{sec:calc_radar_pos}* (Only once if Radar data exists)
\end{itemize}

\paragraph{Creating a New MySQL Database from Verif Export}

\begin{itemize}
 \item \nameref{sec:mysql_connect}
 \item \nameref{sec:mysql_open_db}
 \item \nameref{sec:mysql_import}
 \item \nameref{sec:calc_radar_pos}* (Only once if Radar data was imported)
\end{itemize}

\subsection{Using a Database Created From Import}

\paragraph{Acessing a Previously Created SQLite Database} 

\begin{itemize}
 \item \nameref{sec:sqlite_open}
\end{itemize}

\paragraph{Creating a New SQLite Database}

\begin{itemize}
 \item \nameref{sec:sqlite_open}
\end{itemize}
\ \\

After such steps, data can be imported if wanted (not recommended for Verif databases):

\paragraph{Importing ASTERIX Data} 

\begin{itemize}
 \item \nameref{sec:asterix_import}
 \item \nameref{sec:manage_datasources}*
 \item \nameref{sec:calc_radar_pos}* (If Radar data was imported)
 \item \nameref{sec:calc_artas_assoc}* (If ARTAS data with TRIs was imported)
\end{itemize}

\paragraph{Importing JSON Data} 

\begin{itemize}
 \item \nameref{sec:json_import}
 \item \nameref{sec:manage_datasources}*
 \item \nameref{sec:calc_radar_pos}* (If Radar data was imported)
\end{itemize}

\subsection{Starting}
After such steps, the database is usable and usage can be started using \nameref{sec:startup_starting}.
