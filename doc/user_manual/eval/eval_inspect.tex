
\section{Results Inspection \& Analysis}
\label{sec:eval_inspect}

After an Evaluation has been run, a new Evaluation report is generated, which is named by the evaluation standard used. 
This report is persisted to the database and can be inspected in the 'Reports' tool of the Flight Deck.
It will be opened automatically as soon as the Evaluation run has ended. \\

Please \textbf{note} that at the moment only one Evaluation report can be persisted for each evaluation standard.
Another Evaluation run for a certain standard will overwrite any existing report for this standard. \\

After selection of the report in the Flight Deck, it will present itself as follows.

\begin{figure}[H]
  \hspace*{-2cm}
    \includegraphics[width=18cm,frame]{figures/eval_report_overview.png}
  \caption{Evaluation report: Overview}
\end{figure}

There are several levels of detail in which sub-results can be inspected. \\

The uppermost is the 'Requirements$\rightarrow$Overview$\rightarrow$Results', giving the sector sums for all requirements. 
This is also the main section of any Evaluation report, which will be displayed when the report is selected. \\

The next level of detail are the sector sum details, located in 'Sectors$\rightarrow$Sector Layer Name$\rightarrow$Requirement Group Name$\rightarrow$Requirement Name'. \\

The lowest level are the per-target details, located in 'Targets$\rightarrow$UTN' and the respective per-target results located in 
'Targets$\rightarrow$UTN$\rightarrow$Sector Layer Name$\rightarrow$Requirement Group Name$\rightarrow$Requirement Name'. \\

By default, when single-clicking a row in a table the respective results are shown in the existing Views. When double-clicking, 
a step into the next level of detail is performed (if available) and the respective report section is opened. \\

Navigation can be made more efficient by returning to the last sub-result by using the 'Back' button on the top-left. 
Every section of the report can also be selected by showing the table of contents by clicking the 'Sections' button on the top right.\\

Different requirement results may use different views to display additional result details for deeper inspection:

\begin{itemize}  
\item Geographic View
\begin{itemize}  
\item Overview grid: 2D grid showing all grid cells with at least 1 issue as red, green otherwise
\item Per-target issue display: Mark test target reports with issues with red border, green otherwise
\end{itemize}
\item Histogram Views
\begin{itemize}  
\item Show counts of \#OK and \#NotOK, or numerical distribution of e.g. calculated error/offset
\end{itemize}
\end{itemize}
\ \\

\subsection{Results Overview}

\begin{figure}[H]
  \hspace*{-2cm}
    \includegraphics[width=18cm,frame]{figures/eval_report_overview.png}
  \caption{Evaluation results: Overview}
\end{figure}

Please note that the results are given as example only and are no indication of performance for any system currently in operation. \\

When single-clicking a row, the respective sector requirement results are shown in the existing Views.

\begin{figure}[H]
  \hspace*{-2.5cm}
    \includegraphics[width=19cm]{figures/geo_eval_detection.png}
  \caption{Evaluation Results: Detection in Geographic View}
\end{figure}

\begin{figure}[H]
  \hspace*{-2.5cm}
    \includegraphics[width=19cm]{figures/geo_eval_pos_correct.png}
  \caption{Evaluation Results: Position Correct in Geographic View}
\end{figure}

\begin{figure}[H]
  \hspace*{-2.5cm}
    \includegraphics[width=19cm]{figures/histogram_eval_pos_correct.png}
  \caption{Evaluation Results: Position Correct in Histogram View}
\end{figure}

Please note that currently evaluation result data can not be selected in Histogram Views. This may be improved in one of the next versions. \\

When double-clicking a row in the Overview results table, a step into the respective sector sum details section is performed.

\subsection{Result Sector Details}

\begin{figure}[H]
  \hspace*{-2cm}
    \includegraphics[width=18cm,frame]{figures/eval_results_sec_det_example.png}
  \caption{Evaluation results: Sector Detail Example}
\end{figure}

At the top, there is an overview table giving the details of the calculation results in the respective sector layer and requirement. \\

At the bottom, further result details are listed per-target, sorted in this example by the Probability of Detection (PD). \\

When left clicking a row, the respective target data and result errors are shown in the existing Geographic Views.

\begin{figure}[H]
  \hspace*{-2.5cm}
    \includegraphics[width=19cm]{figures/eval_results_single_geoview.png}
  \caption{Evaluation results: PD Misses in Geographic View}
\end{figure}

When right-clicking a target, several options are shown:

\begin{itemize}  
\item Show full UTN: Load all data from the respective target (not just used reference / test data)
\item Show Surrounding Data: Load all data surrounding the target in position and time (e.g. to find association issues)
\item Target Usage: Edit the target's usage and constraints, as described in section \nameref{sec:eval_report_targets}
\begin{itemize} 
  \item \textbf{Note} that editing target usage will set the report to out-of-date, making a manual refresh necessary to see the changes
\end{itemize}  
\end{itemize}  
\ \\

When double-clicking a row, a step into the respective target details is performed.

\subsection{Result Per-Target Details}

\begin{figure}[H]
  \hspace*{-2cm}
    \includegraphics[width=18cm,frame]{figures/eval_results_target_example.png}
  \caption{Evaluation results: Per-Target Detail Example}
\end{figure}

At the top, there is an overview table giving the details of the calculation results for the target in the respective sector layer and requirement. \\

At the bottom, further result details are listed per-target-report, sorted in this example by time. \\

When single-clicking a row, the respective target data and the respective single result error are shown in the existing Geographic Views.

\begin{figure}[H]
  \hspace*{-2.5cm}
    \includegraphics[width=19cm]{figures/eval_results_pd_single_tr_geoview.png}
  \caption{Evaluation results: Target Single PD Error in Geographic View}
\end{figure}
