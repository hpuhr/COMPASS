\chapter{Installation}
\label{sec:installation}

ATSDB can either be built from source (which exludes the OSG View, but allows development of additional features), or downloaded as an AppImage (binary distributable which includes the OSG View). The recommended option for non-developers is the AppImage. However, there are a few prerequisites that should be taken into consideration.

\section{Prerequisites}

\subsection{Operating System}

The binary will currently only be provided to (somehwat current) Linux 64-bit operating systems. For the following distributions, the AppImage was reported to be working correctly:

\begin{itemize}  
\item Ubuntu 14.04
\item Ubuntu 17.10
\item Linux Mint 18.3
\end{itemize}

For the following distributions, the AppImage was reported to have issues:

\begin{itemize}  
\item CentOS 6.*: Unfortunately the operating system's glibc versions are too old.
\end{itemize}

If you have tried other OS as the ones listed here, it would be appreciated if you could provide feedback about your experiences to \href{mailto:atsdb@gmx.at}{atsdb@gmx.at}.\\

Other OS (e.g. Windows or Mac) will currently not be supported.

\subsection{Recommended Hardware}

The application will perform best on a workstation with at least the following minimum requirements:

\begin{itemize}  
\item CPU with at least 2 physical cores
\item Dedicated NVidia or ATI graphics card
\item 8GB of RAM or more
\end{itemize}

Depending on the loaded datasize, more RAM or a better graphics card might be of advantage. \\\\

A recommended setup could be similiar to:

\begin{itemize}  
\item Intel i5 or better
\item Decent NVidia or ATI graphics card, e.g.
\begin{itemize}  
\item Nvidia Quattro, GT 740 or later, GTX 660 or later
\item Or equivalent ATI GPU models
\item Refer e.g. to \url{http://gpuboss.com/} for benchmarks
\end{itemize}
\item 16GB of RAM or more
\end{itemize}

\subsection{Graphics Cards \& Drivers}
\label{sec:graphics_installation}

Since there exist hundreds of different graphics card types, and several possible drivers for each of them, support will only be given for ATI or Nvidia graphics cards, with their native drivers.

To find out what graphics cards are available in the system, use \textit{lspci} (might have to be installed):

\begin{verbatim}
lspci | grep VGA
\end{verbatim}

Output might look like this:

\begin{verbatim}
01:00.0 VGA compatible controller: NVIDIA Corporation GP104 [GeForce GTX 1080]
\end{verbatim}

To find out which graphics card and driver are being used, the program \textit{glxinfo} can to be used (might have to be installed). Please execute the following command:

\begin{verbatim}
glxinfo | grep OpenGL
\end{verbatim}

The output might look something like this:

\begin{verbatim}
OpenGL vendor string: NVIDIA Corporation
OpenGL renderer string: GeForce GTX 1080/PCIe/SSE2
OpenGL core profile version string: 4.5.0 NVIDIA 384.111
OpenGL core profile shading language version string: 4.50 NVIDIA
OpenGL core profile context flags: (none)
OpenGL core profile profile mask: core profile
OpenGL core profile extensions:
OpenGL version string: 4.5.0 NVIDIA 384.111
OpenGL shading language version string: 4.50 NVIDIA
OpenGL context flags: (none)
OpenGL profile mask: (none)
OpenGL extensions:
OpenGL ES profile version string: OpenGL ES 3.2 NVIDIA 384.111
OpenGL ES profile shading language version string: OpenGL ES GLSL ES 3.20
OpenGL ES profile extensions:
\end{verbatim}

This gives several important pointer:

\begin{itemize}  
\item OpenGL vendor string: This is the grapics card driver. Only NVidia or ATI ones are supported.
\item OpenGL version string: This is the OpenGl version, 3.0 or later is recommended.
\item OpenGL shading language version string: This is the OpenGL shader version, 3.0 or later is recommended.
\end{itemize}

\paragraph{Unsupported Graphics Cards or Drivers}

If different graphics cards or driver are in use, the output of glxinfo might be similiar to this:

\begin{verbatim}
OpenGL vendor string: nouveau
OpenGL renderer string: Gallium 0.4 on NVE6
OpenGL core profile version string: 3.1 (Core Profile) Mesa 9.2.5
OpenGL core profile shading language version string: 1.40
OpenGL core profile context flags: (none)
OpenGL core profile extensions:
OpenGL version string: 3.0 Mesa 9.2.5
OpenGL shading language version string: 1.30
OpenGL context flags: (none)
OpenGL extensions:
\end{verbatim}

Or this:

\begin{verbatim}
OpenGL vendor string: Intel Open Source Technology Center
OpenGL renderer string: Mesa DRI Intel(R) Ivybridge Desktop 
OpenGL core profile version string: 3.3 (Core Profile) Mesa 11.0.6
OpenGL core profile shading language version string: 3.30
OpenGL core profile context flags: (none)
OpenGL core profile profile mask: core profile
OpenGL core profile extensions:
OpenGL version string: 3.0 Mesa 11.0.6
OpenGL shading language version string: 1.30
OpenGL context flags: (none)
OpenGL extensions:
OpenGL ES profile version string: OpenGL ES 3.0 Mesa 11.0.6
OpenGL ES profile shading language version string: OpenGL ES GLSL ES 3.00
OpenGL ES profile extensions:
\end{verbatim}

In such cases, the following issues can be expected and will currently not be addressed:

\begin{itemize} 
\item Application might not even start (OpenGl version error)
\item Slow display performance (OpenGl emulation be CPU-based Mesa driver)
\item Graphical display errors (wrong colours, artefacts, etc) 
\end{itemize} 

Please know that they might also work, but no guarantees or support can be given at this moment.

\section{Using the AppImage}

To summarize, an AppImage is a form of binary distribution in one complete package, which does not require installation of any libraries or alteration of the operating system. Please refer to \url{https://appimage.org/} to get additional information.

To obtain the ATSDB AppImage, download the most current AppImage from \url{https://github.com/hpuhr/ATSDB/releases}. To execute, make it executable using the following command:
\begin{verbatim}
chmod +x ATSDB-x86_64_release.AppImage
\end{verbatim}

Generally speaking, that is it. The application can be run using the following command:
\begin{verbatim}
./ATSDB-x86_64_release.AppImage
\end{verbatim}

The following notes that should be considered:

\begin{itemize}  
\item The AppImage should run under any Linux distribution of similiar date to Ubuntu 14.04 or later, but no guarantees can be made.
\item The AppImage performance is somewhat lessened in comparison to the a later Ubuntu version, which in all probability is just that the graphics system is that outdated. This will be investigated further.
\item OSG View rendering is performed according to the local graphics card and driver, and might be limited by their capabilities.
\end{itemize}

\section{Building from Source Code}
Currently, ATSDB can be installed as source code and compiled by the user. Any recent Linux installation should work without issues. Regarding software version, no particularly new features were used, so older versions than the verified ones should also work. \\

The following software has to be installed on the workstation:

\begin{table}[H]
  \center
  \begin{tabular}{ | l | l | l |}
    \hline
    \textbf{Package} & \textbf{Description} & \textbf{Verified version} \\ \hline
    g++ & C++11 capable C++ compiler & 7.2.0 \\ \hline
    cmake & CMake build tool & 3.9.1 \\ \hline
    qt5-default & Qt5 development package & 5.9.1 \\ \hline
    libboost-dev & Boost development libraries & 1.62.0.1 \\ \hline
    mysql++-dev & MySQL library bindings & 3.2.2 \\ \hline
    libmysqlclient-dev & MySQL development library & 1.0.2 \\ \hline
    libsqlite3-dev & Sqlite3 development files & 3.2.2 \\ \hline
    libgdal-dev & Geospatial data abstraction library & 2.2.1 \\ \hline
    tinyxml2-dev & XML parsing library & 5.0.1 \\ \hline
    log4cpp5-dev & Logging libary & 1.1.1 \\ \hline
    libarchive-dev & Archive libary & 3.2.2 \\ \hline
    libeigen3-dev & Linear algebra libary & 3.3.4 \\ \hline
    libjsoncpp-dev & JSON parsing \& writing library & 1.7.4-3 \\ \hline
    doxygen & Documentation generation & 1.8.13 \\ 
    \hline
  \end{tabular}
  \caption{Required software/libraries}
\end{table}

To install to source code, either download the latest released version from \\ \url{https://github.com/hpuhr/ATSDB/releases} \\
or use the following command (git required) to clone the current repository:

\begin{verbatim}
git clone https://github.com/hpuhr/ATSDB.git
\end{verbatim}

Enter the ATSDB source folder, and execute cmake to create a Makefile:

\begin{verbatim}
cmake .
\end{verbatim}

The output should look like this:
\begin{verbatim}
sk@golem:~/test/ATSDB$ cmake .
  Path: /home/sk/test/ATSDB
-- The C compiler identification is GNU 7.2.0
-- The CXX compiler identification is GNU 7.2.0
-- Check for working C compiler: /usr/bin/cc
-- Check for working C compiler: /usr/bin/cc -- works
-- Detecting C compiler ABI info
-- Detecting C compiler ABI info - done
-- Detecting C compile features
-- Detecting C compile features - done
-- Check for working CXX compiler: /usr/bin/c++
-- Check for working CXX compiler: /usr/bin/c++ -- works
-- Detecting CXX compiler ABI info
-- Detecting CXX compiler ABI info - done
-- Detecting CXX compile features
-- Detecting CXX compile features - done
  System: Linux-4.13.0-16-generic
  Install Path: /home/sk/test/ATSDB/dist
  Platform: Linux
-- Looking for pthread.h
-- Looking for pthread.h - found
-- Looking for pthread_create
-- Looking for pthread_create - not found
-- Looking for pthread_create in pthreads
-- Looking for pthread_create in pthreads - not found
-- Looking for pthread_create in pthread
-- Looking for pthread_create in pthread - found
-- Found Threads: TRUE  
-- Boost version: 1.62.0
-- Found the following Boost libraries:
--   regex
--   system
--   thread
--   chrono
--   date_time
--   atomic
  Boost_INCLUDE_DIR: /usr/include
  Boost_LIBRARY_DIR: 
  CMAKE_MODULE_PATH: /home/sk/test/ATSDB/cmake_modules
-- Found MySQL: /usr/lib/libmysqlpp.so
  MySQLpp_INCLUDE_DIR: /usr/include/mysql++
  MySQLpp_LIBRARY: /usr/lib/libmysqlpp.so
-- Found MySQL: /usr/lib/x86_64-linux-gnu/libmysqlclient.so
  MYSQL_INCLUDE_DIR: /usr/include/mysql
-- Found Sqlite3 header file in /usr/include
-- Found Sqlite3 libraries: /usr/lib/x86_64-linux-gnu/libsqlite3.so
  SQLite3_INCLUDE_DIR: /usr/include
  SQLite3_LIBRARY_DIR: /usr/lib/x86_64-linux-gnu/libsqlite3.so
-- Found Log4CPP
  LOG4CPP_INCLUDE_DIR: /usr/include
  LOG4CPP_LIBRARY: /usr/lib/x86_64-linux-gnu/liblog4cpp.so
-- Found GDAL: /usr/lib/libgdal.so  
  GDAL_INCLUDE_DIRS: /usr/include/gdal
  GDAL_LIBRARIES: /usr/lib/libgdal.so
  TINYXML2_INCLUDE_DIR: /usr/include
  TINYXML2_LIBRARY: /usr/lib/x86_64-linux-gnu/libtinyxml2.so
-- Found Doxygen: /usr/bin/doxygen (found version "1.8.13") found components:
    doxygen 
CPACK_SOURCE_IGNORE_FILES = /CMakeFiles/;/_CPack_Packages/;/dist/;/.git/;
-- Configuring done
-- Generating done
-- Build files have been written to: /home/sk/test/ATSDB
\end{verbatim}

Then, compile the source by executing:

\begin{verbatim}
make
\end{verbatim}

The output should look like this:

\begin{verbatim}
Scanning dependencies of target atsdb_autogen
[  1%] Automatic MOC for target atsdb
Generating MOC predefs moc_predefs.h
Generating MOC source 2MJGWJB4P3/moc_mainwindow.cpp
...
Generating MOC source 3JYSCEOBDA/moc_viewmanagerwidget.cpp
Generating MOC compilation mocs_compilation.cpp
[  1%] Built target atsdb_autogen
Scanning dependencies of target atsdb
[  2%] Building CXX object CMakeFiles/atsdb.dir/src/atsdb.cpp.o
[  2%] Building CXX object CMakeFiles/atsdb.dir/src/buffer/arraylist.cpp.o
...
[ 93%] Linking CXX shared library dist/lib/libatsdb.so
[ 93%] Built target atsdb
Scanning dependencies of target atsdb_client_autogen
[ 94%] Automatic MOC for target atsdb_client
Generating MOC predefs moc_predefs.h
Generating MOC source 2MJGWJB4P3/moc_mainwindow.cpp
...
Generating MOC source 3JYSCEOBDA/moc_viewmanagerwidget.cpp
Generating MOC compilation mocs_compilation.cpp
[ 94%] Built target atsdb_client_autogen
Scanning dependencies of target atsdb_client
...
[100%] Linking CXX executable atsdb_client
[100%] Built target atsdb_client
\end{verbatim}

To install the ATSDB client, execute the following command in the build folder:

\begin{verbatim}
sudo make install
\end{verbatim}

This will install binary and a data folder to your system directories, commonly in '/usr/local/bin/', '/usr/local/lib/' and '/usr/local/atsdb/' (or similiar, depending on you Linux distribution.

To start the ATSDB client execute:

\begin{verbatim}
atsdb_client
\end{verbatim} 
