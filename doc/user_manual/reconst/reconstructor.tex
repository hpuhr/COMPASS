\chapter{Reconstructor}
\label{sec:reconst} 

The reconstructor associates all available target reports to unique targets, each identifed by a \textbf{U}nique \textbf{T}arget \textbf{N}umber (\textbf{UTN}). For each of the targets, reference trajectories ("optimal" position estimates akin to system track updates) are calculated and written to the database. \\

Reference trajectories are calculated based on available sensors and System Trackers, as selected by the user. \\

2 different reconstructors exist, but the advanced reference trajection calculation feature (Probabilistic + IMM Reconstructor) is only available under a commercial professional license. In the free license, a basic reconstructor (Scoring + UMKalman) is available. Please refer to \nameref{sec:ui_configure_licenses} for licensing information. \\

\begin{itemize}  
   \item Basic: Scoring + UMKalman
   \begin{itemize}  
   \item Included in free version
   \item Association based on secondary attributes and simple, configurable distance scoring
   \item Position estimation using Linear Uniform Motion Kalman filter
   \end{itemize}  
   \item Advanced: Probabilistic + IMM
   \begin{itemize}  
   \item Requires commercial professional license
   \item Association based on secondary attributes and self-adaptive probabilistic thresholds
   \item Uses Radar bias estimation \& correction, ADS-B geometric altitude for slant range correction, accuracy -re-estimation and scaling, ADS-B verification, ...
   \item Position estimation using Interacting Multiple Models (IMM) filter, with a Rauch-Tung-Striebel (RTS) smoother
   \end{itemize}
\end{itemize}  

For optimal performance, the processing of the target reports and reconstruction is time-sliced. In (commonly) 15 minute intervals, all target reports are loaded, associated to UTNs, references are calculated and written to the database, after which the next time-slice is loaded. \\

In the 'Probabilistic + IMM' reconstructor, each data slice is processed multiple times, to re-estimate correctable bias' as well as rescale target report accuracies. \\

\subfile{reconstructor_config}

\section{Workflow}
For each processed slice, the following general steps are performed:

\begin{itemize}
\item Load all target reports for the current time-slice
\item Remove all target reports / calculated information for the previous time-slice
\item Load all target reports, per data source
\item Associate target reports to UTNs
   \begin{itemize}  
   \item If possible, associate by matching reliable secondary attribute
    \begin{itemize}  
    \item Mode S Address, Mode S Identification, Reliable Track Number
    \end{itemize}
   \item Otherwise, associate by matching Mode A/C and position
   \item Otherwise, associate by position only
   \end{itemize}
\item Self-associate new targets (find matching UTNs created by the same target)
\item Retry-Associate target reports
\item Compute references
\item Calculate statistics based on references
\item Write references \& associations
\end{itemize}
\ \\

There are several algorithmic features of interest, which are described in the following chapters, but deeper discussion is outside the scope of this document.

\subsection{Associate Target Reports to Targets}

The following steps are performed for each target report:
\begin{itemize}
\item Check if association can be made using lookup lists of targets based on
\begin{itemize}
\item Mode S address
\item Mode S identification
\begin{itemize}
\item Ignoring "00000000", "????????", "        " values
\end{itemize}
\item CAT062/Reference Trajectory Track number (data source + line specific)
\end{itemize}
\item Otherwise, associate by matching Mode A/C and position
\item Otherwise, associate by position only
\item If a matching target is found, the best match is used for association
\item If no matching target is found
\begin{itemize}
\item If the target report has Mode S address/identification or a reliable Track Number, a new target is created
\item Otherwise no association is made (stored for later retry-association)
\end{itemize}
\ \\

If a match can be made, the match is verified - e.g. the matching based on Track Number can be changed if e.g. the time difference to the last target update is larger than the 'Maximum Track Time Difference' parameter, or of the distance threshold is larger than the 'Track Number Disassoc. Distance Factor', assuming the 'Do Track Number Disassoc. Using Distance' is set. \\

If still valid, the target report is associated to the specific UTN, which in turn updates the respective lookup lists.

\subsubsection{Position Matching}

For considering a position match, a position score is calculated for each target report, to all possible targets. In the case of 'Scoring + UMKalman', the distance offsets (target report to interpolated target position at time of target report) are compared to the 'Maximum Acceptable Offset', and only associated to the best match if within such a distance.

\subsubsection{New Target Creation}

When creating a new target using a target report, the secondary attributes are stored in the lookup lists of targets.

\subsection{Self-associate New Targets}

For all newly created targets, matching other targets are checked if a positive score would allow merging the target pair. This allows merging created targets even if single target reports (e.g. during the somewhat unstable beginning of tracking) did not allow for it.

\begin{itemize}
\item Associate new per-source targets to existing targets based on
\begin{itemize}
\item Minimum time overlap: 'Minimum Time Overlap Probability' parameter
\item Mode A code(s) similarity
\item Mode C code(s) similarity: 'Maximum Altitude Difference' parameter
\item Position similarity
\begin{itemize}
\item Count number of updates being in different distances: \#Erroneous, \#Dubious, \#Acceptable
\item Minimum number of updates checked: 'Minimum Updates' parameter
\item Associate if probability of erroneous and dubious are not too high
\end{itemize}
\end{itemize}
\end{itemize}
\ \\

\subsection{Discussion}

The user should be aware that, while this association feature is quite capable, it is still somewhat limited. It strongly depends on the correctness of Mode S addresses/identification, as well as the Tracker information (track number, secondary information and position information). If the mentioned information is erroneous, the made association will be sub-optimal or plainly wrong. \\

Also, for association of non-Mode S target reports a trade-off has to be made in the 'Maximum Acceptable Distance' parameter, especially if they are primary-only. It should be set within the limits of Reference/Tracker error plus maximum sensor error (which can still include radar bias') and the used target separation minima. This is of course not well-suited for strongly different sensors accuracies and separations (e.g. when mixing ground and air surveillance data). \\

\section{Probabilistic + IMM Features}

Without going into detail, in this section a short overview is given over the capabilities of the advanced reconstructor. The main improvements are:

\begin{itemize}
\item Rescaling accuracies
\begin{itemize}
\item Assess / verify accuracy of tracked data sources
\item Assess / verify accuracy of ADS-B position quality indicators, per transponders
\item Estimate radar plot position accuracy
\item Using parameters based models as well as 2D accuracy maps
\end{itemize}
\item Correct Radar bias': Estimate and correct azimuth bias, range bias \& gain
\item Use ADS-B geometric altitude for slant-range correction
\begin{itemize}
\item Method based on published paper: \href{https://doi.org/10.3390/engproc2022028008}{'Usage of Geometric Altitude for Radar Plot Position Improvements' by H. Puhr, MDPI 2022}
\item Use geometric altitudes of ADS-B data for improved slant-range correction (over using barometric altitudes)
\end{itemize}
\item Do position outlier checks: Associate but do not use outlier positions
\begin{itemize}
\item Based on estimated target position estimate accuracy and (re-scaled) target report accuracy
\end{itemize}
\item Use Risky ADS-B: (disabled by default) Estimate ADS-B accuracy of transponders without usable position quality indicators
\begin{itemize}
\item Only usable if 'Rescaling Accuracies' is enabled
\end{itemize}
\item Generally: Do not use static position thresholds, but probabilistic ones
\item System Tracker Spline interpolation: Generate 1-second updates to compensate for high-update-rate data sources
\begin{itemize}
\item Mahalanobis distances based on accuracy estimates
\end{itemize}
\item Improved reconstruction algorithms for reference reconstruction
\begin{itemize}
\item \textbf{I}nteracting \textbf{M}ultiple \textbf{M}odels (IMM) filter
\item \textbf{R}auch-\textbf{T}ung-\textbf{S}triebel (RTS) smoother
\end{itemize}
\end{itemize}

\subsection{Radar Plot Position Improvements}

Using the features mentioned above, the 'Correct Radar Bias' and the 'Use ADS-B geometric altitude for slant-range correction' features can result in the following real-life position improvements: median org (original) error vs. median cor (corrected) error in [meters], for 10 different Radars:

\begin{lstlisting}
Rad1 estimateXYStdDev: median error
	 org avg 70.29 mad 34.20
	 cor avg 65.18 mad 32.13 cnt 6476
Rad2 estimateXYStdDev: median error
	 org avg 59.40 mad 31.81
	 cor avg 48.79 mad 29.69 cnt 22571
Rad3 estimateXYStdDev: median error
	 org avg 136.98 mad 57.03
	 cor avg 65.69 mad 34.90 cnt 25725
Rad4 estimateXYStdDev: median error
	 org avg 73.16 mad 33.48
	 cor avg 62.76 mad 34.57 cnt 26050
Rad5 estimateXYStdDev: median error
	 org avg 102.84 mad 54.71
	 cor avg 95.01 mad 61.07 cnt 27595
Rad6 estimateXYStdDev: median error
	 org avg 65.04 mad 34.69
	 cor avg 63.29 mad 38.70 cnt 30520
Rad7 estimateXYStdDev: median error
	 org avg 120.52 mad 57.49
	 cor avg 60.03 mad 33.67 cnt 34208
Rad8 estimateXYStdDev: median error
	 org avg 132.39 mad 88.92
	 cor avg 123.45 mad 79.81 cnt 33445
Rad9 estimateXYStdDev: median error
	 org avg 114.86 mad 56.14
	 cor avg 75.72 mad 46.94 cnt 35744
Rad10 estimateXYStdDev: median error
	 org avg 71.64 mad 32.46
	 cor avg 47.39 mad 24.76 cnt 39444
\end{lstlisting}

\subsubsection{Running}

To run the task, click the 'Run' button. After the assocations are saved, the task is done:


Currently the following sensor data can be incorporated into reference track estimation:

\begin{itemize}
    \item System Tracks
    \item ADS-B \\
\end{itemize}

It is possible to exclude target reports from the estimation based on various criteria. 
The used reconstructor then integrates positions, velocities and any associated uncertainties of 
added target reports, in order to calculate reference positions with estimated locations, velocities and uncertainties. \\

After reference computation the generated reference trajectories for each target will be added to the database, 
using a new data source. \\

The task's dialog will show as follows. \\

% \begin{figure}[H]
%     \center
%       \includegraphics[width=14cm]{figures/ui_task_references_dialog.png}
%     \caption{Calculate References Task - Configuration Dialog}
% \end{figure}

It consists of several configuration tabs:

\begin{itemize}
    \item \textbf{Input Data Sources}: Configuration of data sources to use for reference computation.
    \item \textbf{Position Data Filter}: Filtering options for excluding certain target reports from reference computation.
    \item \textbf{Kalman Settings}: Configuration of the Kalman reconstructor.
    \item \textbf{Output}: Configuration of the generated data source. \\
\end{itemize}

These configuration tabs will be described in more detail below.

\subsubsection{Input Data Sources}

% \begin{figure}[H]
%     \center
%       \includegraphics[frame,width=14cm]{figures/ui_task_references_tab_inputds.png}
%     \caption{Calculate References Task - Input Data Sources Tab}
% \end{figure}

In this dialog the data sources which are incorporated into reference trajectory estimation can be configured.
There exists a checkbox for each possible DSType and then a nested list of checkboxes for individual data sources. \\

\textbf{Note}: At least one data source has to be selected in order to run reference computation.

\subsubsection{Position Data Filter}

% \begin{figure}[H]
%     \center
%       \includegraphics[frame,width=14cm]{figures/ui_task_references_tab_filter.png}
%     \caption{Calculate References Task - Position Data Filter Tab}
% \end{figure}

In this dialog various filters can be enabled in order to remove certain target reports from reference computation. 
The following tables describe the existing options.

\begin{table}[H]
    \center
    \begin{tabularx}{\textwidth}{ | l | l | X |}
        \hline
        \textbf{Parameter} & \textbf{Default} & \textbf{Description} \\ \hline
        Filter Position Data & enabled & Enables/disables filtering of position data \\ \hline
    \end{tabularx}
\end{table}

\textbf{Note}: All other options will be ignored if this option is disabled. \\

\textit{Tracker Position Data Usage}:
\begin{table}[H]
    \center
    \begin{tabularx}{\textwidth}{ | l | l | X |}
        \hline
        \textbf{Parameter} & \textbf{Default} & \textbf{Description} \\ \hline
        Only Use Confirmed & enabled & Use only non-tentative track updates \\ \hline
        Only Use Non-Coasting & enabled & Use only non-coasting track updates \\ \hline
        Only Use Detected Report & disabled & Use only non-zero Detection Type track updates \\ \hline
        Only Use Non-Single PSR-only Detections & disabled & Use no mono PSR-only track updates \\ \hline
        Only Use High Accuracy & enabled & Use only track updates with Std.Dev. smaller than threshold \\ \hline
        Minimum Position Stddev [m] & 30 & Std.Dev. threshold \\ \hline
    \end{tabularx}
\end{table}

% \textit{ADS-B Position Data Usage}:
% \begin{table}[H]
%     \center
%     \begin{tabularx}{\textwidth}{ | l | l | X |}
%         \hline
%         \textbf{Parameter} & \textbf{Default} & \textbf{Description} \\ \hline
%         Only Use MOPS V1 / V2 & enabled & Use only MOPS V1 and V2 ADS-B data \\ \hline
%         Only Use High NUCp / NIC & disabled & Use only ADS-B data larger than NUCp / NIC threshold \\ \hline
%         Minimum Position NUCp / NIC & 4 &  NUCp / NIC threshold \\ \hline
%         Only Use High NACp & enabled & Use only ADS-B data larger than NACp threshold \\ \hline
%         Minimum Position NACp & 4 & NACp threshold \\ \hline
%         Only Use High SIL & disabled & Use only ADS-B data larger than SIL threshold \\ \hline
%         Minimum Position SIL & 1 & SIL threshold \\ \hline
%     \end{tabularx}
% \end{table}

\subsubsection*{Kalman Settings}

% \begin{figure}[H]
%     \center
%       \includegraphics[frame,width=14cm]{figures/ui_task_references_tab_kalman.png}
%     \caption{Calculate References Task - Kalman Settings Tab}
% \end{figure}

In this dialog the used reference trajectory reconstructor can be chosen and configured.
The following tables describe the existing options.

\begin{table}[H]
    \center
    \begin{tabularx}{\textwidth}{ | l | l | X |}
        \hline
        \textbf{Parameter} & \textbf{Default} & \textbf{Description} \\ \hline
        Reconstructor & UMKalman2D & Reconstructor type used for reference computation \\ \hline
    \end{tabularx}
\end{table}

The currently implemented reconstructors are based on the well-known Kalman filter and its various variants.
Currently the following 'flavours' are implemented. \\

\begin{itemize}
    \item \textbf{Uniform Motion Kalman (UMKalman2D)}: 
        Classic Kalman filter assuming a constant velocity between consecutive measurements
    \item \textbf{Accelerated Motion Kalman (AMKalman2D)}: 
        Unscented Kalman filter assuming constant acceleration between consecutive measurements \\
\end{itemize}

\textit{Default Uncertainties}: Parameters related to the uncertainties used in the Kalman filter, in addition to those provided 
by the input data.
\begin{table}[H]
    \center
    \begin{tabularx}{\textwidth}{ | l | l | l | X |}
        \hline
        \textbf{Parameter} & \textbf{Default} & \textbf{Unit} & \textbf{Description} \\ \hline
        %Measurement Stddev & 30 & m & Default standard deviation for added measurements \\ \hline
        %Measurement Stddev (high) & 1000 & m & Default high standard deviation for added measurements, 
        %    used if important data is not provided by the data base (e.g. velocity) \\ \hline
        Process Stddev & 30 & m & Process noise standard deviation of the modelled Kalman process \\ \hline
        Tracker Velocity Stddev & 50 & m & Default standard deviation for tracker velocities, 
            used if not provided by the data \\ \hline
        %Tracker Acceleration Stddev & 50 & m & Default standard deviation for tracker accelerations,
        %    used if not provided by the data \\ \hline
        ADS-B Velocity Stddev & 50 & m & Default standard deviation for ADS-B velocities, 
            used if not provided by the data \\ \hline
        %ADS-B Acceleration Stddev & 50 & m & Default standard deviation for ADS-B accelerations, 
        %    used if not provided by the data \\ \hline
    \end{tabularx}
\end{table}

\textbf{Note}: When using the default settings, target reports without position accuracy data are not used for reference position estimation. However, if such filters (in the \textit{Position Data Filter} tab) are disabled, if no position accuracy is provided by the data for a specific target report, a standard value of 100 meters is automatically assumed. \\

Please also note that if an important value such as velocity is not provided by the data, a value of zero with a very high standard deviation of 1000 meters is automatically assumed. \\

\textit{Chain Generation}: Parameters related to criteria causing the reconstructor to reinitialize the Kalman filter, 
resulting in the reference trajectory of a track being split into multiple sub-chains.
\begin{table}[H]
    \center
    \begin{tabularx}{\textwidth}{ | l | l | l | X |}
        \hline
        \textbf{Parameter} & \textbf{Default} & \textbf{Unit} & \textbf{Description} \\ \hline
        Minimum Time Step & 0 & s & Minimum time difference between consecutive measurements (measurements obtaining a 
            lower time difference to their predecessor will be skipped) \\ \hline
        Maximum Time Step & 11 & s & Maximum time difference between consecutive measurements (measurements obtaining a 
            higher time difference to their predecessor will cause the Kalman filter to reset, 
            resulting in a new sub-chain) \\ \hline
        Minimum Chain Size & 2 & & Minimum number of created reference positions in a sub-chain, in order to add the chain 
            to the final result \\ \hline
    \end{tabularx}
\end{table}

\textit{Additional Options}:
\begin{table}[H]
    \center
    \begin{tabularx}{\textwidth}{ | X | l | l | X |}
        \hline
        \textbf{Parameter} & \textbf{Default} & \textbf{Unit} & \textbf{Description} \\ \hline
        Smooth Results & enabled & & Smooth the results of the Kalman filter using a Rauch–Tung–Striebel smoother \\ \hline
        Resample System Tracks & enabled & & Resample input system tracks using cubic spline interpolation \\ \hline
        Resample System Tracks - Resample Interval & 1 & s & Time interval at which system tracks are resampled \\ \hline
        Resample System Tracks - Maximum Time Step & 30 & s & Maximum time difference between consecutive tracker 
            reports in order to resample the segment \\ \hline
        Resample Result & enabled & & Resample the generated reference trajectories by interpolating the final Kalman states
            at a fixed rate \\ \hline
        Resample Result - Resample Interval & 2 & s & Time interval at which the result trajectories are resampled \\ \hline
        Resample Result - Resample Process Stddev & 10 & m & Standard deviation of the Kalman process when interpolating the 
            Kalman states during result resampling \\ \hline
    \end{tabularx}
\end{table}

Resampling of system tracks can be enabled in order to obtain similar sample rates for tracker and ADS-B
data, which reduces a sensor-specific bias of the Kalman filter. Cubic B-splines are used to interpolate the tracker target
reports in a relaxed way. In case a spline segment deviates too strongly from its interval, a linear interpolation is 
used instead. \\

Result resampling can be enabled in order to obtain reference trajectories with a homogeneous sample rate. 
It is carried out by interpolating the Kalman states at fixed time-steps. The accuracy of the resulting samples
follows the Kalman accuracies more closely when interpolating near the endpoints of a segment, and it will decrease
in the middle of a segment. The amount of decrease of accuracy will depend on the time duration of the segment and the used 
process standard deviation (parameter \textit{Resample Process Stddev}).

\subsubsection{Output}

% \begin{figure}[H]
%     \center
%       \includegraphics[frame,width=14cm]{figures/ui_task_references_tab_info.png}
%     \caption{Calculate References Task - Output Tab}
% \end{figure}

Here the data source created by reference computation can be configured. 
Reference data will be stored to the database as part of this data source.
The following options are available. \\

\textit{Output Data Source}:
\begin{table}[H]
    \center
    \begin{tabularx}{\textwidth}{ | l | l | X |}
        \hline
        \textbf{Parameter} & \textbf{Default} & \textbf{Description} \\ \hline
        Name & CalcRefTraj & Name of the created data source \\ \hline
        SAC & 0 & SAC of the created data source \\ \hline
        SIC & 1 & SIC of the created data source \\ \hline
        Line ID & 1 & Line ID under which the created data is stored \\ \hline
    \end{tabularx}
\end{table}

Please \textbf{note} that an already existing reference data source is just updated by running the 
reference computation again. This way it is possible to store several reference trajectories to 
different line IDs for the sake of comparison.

\subsubsection{Running The Task}

If configured correctly, the task can be started by clicking the \textit{Run} button.
The current configuration can always be stored without running the task by clicking the \textit{Close} button. 
This can be useful when testing the configuration for using \nameref{sec:ui_proc_calc_references_preview}. \\

When the task is started, the following progress dialog is shown to the user.

% \begin{figure}[H]
%     \center
%       \includegraphics[width=9cm]{figures/ui_task_references_dialog_result.png}
%     \caption{Calculate References Task - Calculate References Status Dialog}
% \end{figure}

First the elapsed time and the current status of the reconstruction process are shown.
Then a few sections give more inside into what happened in the various stages of the process, 
which are described in more detail below.

\paragraph{Loaded Data} This section gives insight into to what ratio data has been loaded from
the database. It shows total and currently loaded data counts, and loading percentage, broken down into
individual DSTypes.

\paragraph{Used Position Data} This section gives insight into to what ratio loaded data has been
included in the reconstruction process after position filtering. 
It shows total data counts, and the amount of data items used for reconstruction in absolute numbers and percentages,
broken down into individual DSTypes.

\paragraph{References Info} This section displays information about the reference computation result.
It shows the number of reference trajectories created, the total number of created target reports, and 
accuracy statistics, which can be used to quantify how well the reconstructor performed.

After finishing, the progress dialog may also display error information. Reference computation can mainly fail in two ways. \\

Badly configured data sources or non-optimal filter settings may result in no data going into the reference computation pipeline.
This will result in the error message \textit{'No input data for reference computation'}. In this case it may help to 
check the included data sources and the chosen filter settings in the reference computation configuration (see above). \\

In case the reconstructor fails to generate any reference data for the included data, the error 
\textit{'No reference data created'} will be shown. A non-optimal Kalman configuration might be the cause for this.

\subsubsection{Reconstruction Preview}
\label{sec:ui_proc_calc_references_preview}

After selecting a desired set of parameters, it can be tiresome to run reference computation for the entire 
set of targets in order to obtain a feedback on the quality of the result. In order to get a faster feedback on the 
chosen parameter set, a \textit{Reconstruction Preview} can be executed for a single chosen track. 
Unlike reference computation for all tracks, the computed reference trajectory will not be written to the database.
Instead it will be visualized in an open Geographic View as an annotation, along with additional intermediate data.
This intermediate data can be especially useful to inspect the various stages of the Kalman process. \\

For this feature, first a set of parameters can be selected in the \textit{Calculate References} dialog, as described above.
Then the current settings are stored by closing the dialog again using the \textit{Close} button on the bottom left. \\

Next, a desired Target can be selected in the \textit{Targets} tab of the main window (see \nameref{sec:ui_targets}).
The reconstruction preview can be started by right-clicking on the selected target and choosing \textit{Reconstruction Preview}.

% \begin{figure}[H]
%     \center
%       \includegraphics[frame,width=16cm]{figures/ui_task_references_recprev_start.png}
%     \caption{Starting a Reconstruction Preview} 
% \end{figure}

After reconstruction has finished, the results can be inspected in the \textit{Annotations} node of Geographic View's \textit{Layer} tab.

% \begin{figure}[H]
%     \center
%       \includegraphics[frame,width=16cm]{figures/ui_task_references_recprev_anno.png}
%     \caption{Reconstruction Preview - Annotations} 
% \end{figure}

Up to three trajectories are shown in different colors, each of them obtaining positions, speed vectors and error ellipses,
and each of them representing different stages of the reconstruction process.

\paragraph{Kalman} Shows the estimates coming directly out of the Kalman filter.
Using this annotation the performance of the Kalman filter can be visually inspected.

% \begin{figure}[H]
%     \center
%       \includegraphics[width=10cm]{figures/ui_task_references_recprev_kalman.png}
%     \caption{Reconstruction Preview - Annotation \textit{Kalman}} 
% \end{figure}

\paragraph{Kalman (smoothed)} Shows the result of the RTS smoother (if smoothing enabled).
Using this annotation the performance of the RTS smoother can be visually inspected.

% \begin{figure}[H]
%     \center
%       \includegraphics[width=10cm]{figures/ui_task_references_recprev_smoothed.png}
%     \caption{Reconstruction Preview - Annotation \textit{Kalman (smoothed)}} 
% \end{figure}

\paragraph{Kalman (resampled)} Shows the homogeneously resampled Kalman states (if result resampling enabled).
Using this annotation result resampling can be visually inspected.

% \begin{figure}[H]
%     \center
%       \includegraphics[width=10cm]{figures/ui_task_references_recprev_final.png}
%     \caption{Reconstruction Preview - Annotation \textit{Kalman (resampled)}} 
% \end{figure}

Additionally, an annotation called \textit{Interpolated Tracks} shows the spline-resampled system tracks in grey color,
if system track resampling is enabled.
% 
% \begin{figure}[H]
%     \center
%       \includegraphics[width=10cm]{figures/ui_task_references_recprev_interp.png}
%     \caption{Reconstruction Preview - Annotation \textit{Interpolated Tracks}} 
% \end{figure}
