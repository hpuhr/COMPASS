\chapter{ScatterPlot View}
\label{sec:scatter_view}

A ScatterPlot View displays the distribution of two numerical variables as points. When started, it presents itself in the following manner.

\begin{figure}[H]
    \hspace*{-2cm}
    \includegraphics[width=18cm,frame]{figures/scatter_start.png}
  \caption{ScatterPlot View startup}
\end{figure}

\section{Layout}

On the left side resides the plot area in which the data is visualized (if data has been loaded). The tool bar at the top shows the currently selected tool and the available actions.\\

On the right side resides the configuration area, which allows configuring what data is loaded and how it is displayed. The 'Reload' button on the bottom can be used to trigger a reload of the view's data.\\

Both areas can be resized and hidden if desired.

\section{Data Loading}

To load the data the mechanism described in section \nameref{sec:ui_overview} or the 'Reload' button can be used. To filter the dataset, the mechanism described in section \nameref{sec:filters} can be used. \\

\begin{figure}[H]
    \hspace*{-2cm}
    \includegraphics[width=18cm,frame]{figures/scatter_loaded.png}
  \caption{ScatterPlot View after loading data}
\end{figure}

The values of the selected variables are used for positioning on the x-axis and y-axis. For each value pair a data point is generated. 
In the current example the WGS-84 meta-variables 'Longitude' and 'Latitude' are used, showing a similar view as the Geographic View. \\

Data points are grouped and colored by the data source and line ID they belong to. 
The data points belonging to a certain line ID of a data source thus form an individual scatter data item with an assigned unique color.

%TODO_V7 i think here it would be immensely important to tell the user why he might not see any data at all when selecting variables from two different DBContents!

%On the right side of the plot a legend is shown, giving the total counts of all data points.

\section{Usage}

\subsection{Toolbar}

The first buttons can be used to switch between the available tools (shortcut refers to keyboard shortcut).

Note that an active tool can always be ended by pressing the 'Escape' key. 

\begin{table}[H]
  \center
  \begin{tabular}{ | l | l | l | l |}
    \hline
    \textbf{Icon} & \textbf{Shortcut} & \textbf{Text} &  \textbf{Description} \\ \hline
    \includegraphics[width=0.5cm,frame]{../../data/icons/select_action.png} & S & Select & Allows data selection \& de-selection \\ \hline
    \includegraphics[width=0.5cm,frame]{../../data/icons/zoom_select_action.png} & R & Zoom to Rectangle & Allows zooming to the selected rectangle \\ \hline
    
  \end{tabular}
  \caption{Toolbar: Available tools}
\end{table}

The others provide general actions by which the view can be modified.

\begin{table}[H]
  \center
  \begin{tabular}{ | l | l | l | l |}
    \hline
    \textbf{Icon} & \textbf{Shortcut} &\textbf{Text} &  \textbf{Description} \\ \hline
    \includegraphics[width=0.5cm,frame]{../../data/icons/select_invert.png} & & Invert Selection & Selects all de-selected \& vice versa \\ \hline
    \includegraphics[width=0.5cm,frame]{../../data/icons/select_delete.png} & & Delete Selection & De-selects all target reports \\ \hline
    \includegraphics[width=0.5cm,frame]{../../data/icons/zoom_home.png} & Space & Zoom to Home & Pans/zooms to show all existing data \\ \hline
  \end{tabular}
  \caption{Toolbar: Available actions}
\end{table} 

\subsection{Config Tab}

The selection controls on the top define which data variables are used to generate data points for the x/y-axis. 
Such a variable can be any numerical variable. A reload operation might be required for the selection to take effect. \\

If available, plottable scatter data of active View Points can be visualized by checking the 'Show Annotations' box. 
If multiple scatter annotations are present in a View Point, these individual plots can be selected using the 'Plot' combo box.
Plottable annotations might further be grouped into so-called plot groups, which are listed under 'Plot Group'.
The plot group can be switched in case multiple plot groups are available. \\

Below the variable selection, there is a list of all currently displayed scatter data items, shown with their respective color, name and data count.
Individual data items can be shown or hidden by using the check boxes. All items can be hidden simultaneously by pressing the 'Deselect All' button. \\

Scatter data items can be drawn with their data points connected by lines by checking the 'Use Connection Lines' box.
\textbf{Note} that on a reload, if the total number of drawn data points exceeds 100000, connections lines will be deactivated automatically
in order to prevent performance issues. After the reload the connection lines can be reactivated manually by the user if desired.

\subsection{Scatterplot}

\subsubsection{Selection Tool}

If the 'Select' tool is active, data can be selected. Using the left mouse-button a red selection rectangle can be spanned across all data points that should be selected.

\begin{figure}[H]
    \hspace*{-2cm}
    \includegraphics[width=18cm,frame]{figures/scatter_select.png}
  \caption{ScatterPlot View data selection}
\end{figure}

The selected data is then presented in an extra 'Selected' item in the scatter data list.

\begin{figure}[H]
    \hspace*{-2cm}
    \includegraphics[width=18cm,frame]{figures/scatter_selected.png}
  \caption{ScatterPlot View data selected}
\end{figure}

This enables selection of parts of the data based on the presented variables, allowing deeper analysis e.g. of dubious data. \\

The 'Invert Selection' \includegraphics[width=0.5cm,frame]{../../data/icons/select_invert.png} or 'Delete Selection' \includegraphics[width=0.5cm,frame]{../../data/icons/select_delete.png} actions 
allow for easier selection of the wanted target reports. \\

By pressing the 'Control' key while selecting, the newly selected data is added to any previous selection. This can be used to select data incrementally, making more complex selections possible.

\subsubsection{Zoom to Rectangle Tool}

Using the 'Zoom to Rectangle' tool, the left mouse-button can be used to select a rectangular region, to which a zoom operation is performed. \\

\subsubsection{General Zoom}

The mouse wheel can be used to zoom in or out of the presented data, the 'Space' key can be used to reset to the default zoom level (equivalent to \includegraphics[width=0.5cm,frame]{../../data/icons/zoom_home.png}).
